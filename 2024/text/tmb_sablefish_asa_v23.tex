% Options for packages loaded elsewhere
\PassOptionsToPackage{unicode}{hyperref}
\PassOptionsToPackage{hyphens}{url}
%
\documentclass[
]{article}
\usepackage{amsmath,amssymb}
\usepackage{lmodern}
\usepackage{iftex}
\ifPDFTeX
  \usepackage[T1]{fontenc}
  \usepackage[utf8]{inputenc}
  \usepackage{textcomp} % provide euro and other symbols
\else % if luatex or xetex
  \usepackage{unicode-math}
  \defaultfontfeatures{Scale=MatchLowercase}
  \defaultfontfeatures[\rmfamily]{Ligatures=TeX,Scale=1}
  \setmainfont[]{Times New Roman}
\fi
% Use upquote if available, for straight quotes in verbatim environments
\IfFileExists{upquote.sty}{\usepackage{upquote}}{}
\IfFileExists{microtype.sty}{% use microtype if available
  \usepackage[]{microtype}
  \UseMicrotypeSet[protrusion]{basicmath} % disable protrusion for tt fonts
}{}
\makeatletter
\@ifundefined{KOMAClassName}{% if non-KOMA class
  \IfFileExists{parskip.sty}{%
    \usepackage{parskip}
  }{% else
    \setlength{\parindent}{0pt}
    \setlength{\parskip}{6pt plus 2pt minus 1pt}}
}{% if KOMA class
  \KOMAoptions{parskip=half}}
\makeatother
\usepackage{xcolor}
\usepackage[margin=1in]{geometry}
\usepackage{graphicx}
\makeatletter
\def\maxwidth{\ifdim\Gin@nat@width>\linewidth\linewidth\else\Gin@nat@width\fi}
\def\maxheight{\ifdim\Gin@nat@height>\textheight\textheight\else\Gin@nat@height\fi}
\makeatother
% Scale images if necessary, so that they will not overflow the page
% margins by default, and it is still possible to overwrite the defaults
% using explicit options in \includegraphics[width, height, ...]{}
\setkeys{Gin}{width=\maxwidth,height=\maxheight,keepaspectratio}
% Set default figure placement to htbp
\makeatletter
\def\fps@figure{htbp}
\makeatother
\setlength{\emergencystretch}{3em} % prevent overfull lines
\providecommand{\tightlist}{%
  \setlength{\itemsep}{0pt}\setlength{\parskip}{0pt}}
\setcounter{secnumdepth}{-\maxdimen} % remove section numbering
\usepackage{float} \floatplacement{figure}{H} \usepackage{fontspec} \setmainfont{Times New Roman}
\usepackage{booktabs}
\usepackage{longtable}
\usepackage{array}
\usepackage{multirow}
\usepackage{wrapfig}
\usepackage{float}
\usepackage{colortbl}
\usepackage{pdflscape}
\usepackage{tabu}
\usepackage{threeparttable}
\usepackage{threeparttablex}
\usepackage[normalem]{ulem}
\usepackage{makecell}
\usepackage{xcolor}
\ifLuaTeX
  \usepackage{selnolig}  % disable illegal ligatures
\fi
\IfFileExists{bookmark.sty}{\usepackage{bookmark}}{\usepackage{hyperref}}
\IfFileExists{xurl.sty}{\usepackage{xurl}}{} % add URL line breaks if available
\urlstyle{same} % disable monospaced font for URLs
\hypersetup{
  pdftitle={Northern Southeast Inside Subdistrict Sablefish (Anoplopoma fimbria) Management Plan and Stock Assessment for 2023},
  pdfauthor={Phil Joy; Rhea Ehresmann},
  hidelinks,
  pdfcreator={LaTeX via pandoc}}

\title{Northern Southeast Inside Subdistrict Sablefish (\emph{Anoplopoma
fimbria}) Management Plan and Stock Assessment for 2023}
\author{Phil Joy\footnote{Alaska Department of Fish and Game, Commercial
  Fisheries Division, Juneau, Alaska} \and Rhea Ehresmann\footnote{Alaska
  Department of Fish and Game, Commercial Fisheries Division, Sitka,
  Alaska}}
\date{June 22, 2023}

\begin{document}
\maketitle

{
\setcounter{tocdepth}{2}
\tableofcontents
}
\begin{center}\rule{0.5\linewidth}{0.5pt}\end{center}

\hypertarget{abstract}{%
\section{ABSTRACT}\label{abstract}}

This report provides an overview of the stock assessment, harvest
strategy, and regulations effective for the 2023 Northern Southeast
Inside (NSEI) sablefish (\emph{Anoplopoma fimbria}) commercial fishery.
The NSEI sablefish commercial fishery is scheduled to open August 15 and
close November 15 and open to both longline and pot gear. The 2023 NSEI
sablefish commercial fishery annual harvest objective is 1,395,686 round
pounds and is based on decrements from an acceptable biological catch of
1,573,109 round pounds. The annual harvest objective is allocated to 73
limited entry Commercial Fisheries Entry Commission C61A permits through
an equal quota share (EQS) system, resulting in a 2023 EQS of 19,121
round pounds for each permit holder.

Key words: sablefish, black cod, \emph{Anoplopoma fimbria}, stock
assessment, annual harvest objective, AHO, catch per unit effort, CPUE,
Northern Southeast, Chatham Strait, NSEI, mark--recapture, tagging

\hypertarget{introduction}{%
\section{INTRODUCTION}\label{introduction}}

The Alaska Department of Fish and Game (ADF\&G) evaluates stock status
and establishes the Northern Southeast Inside (NSEI) acceptable
biological catch (ABC) and subsequent annual harvest objective (AHO).
The NSEI Subdistrict management area (Figure @ref(fig:nseimap)) consists
of all waters as defined in 5 AAC 28.105(a)(2).

The recommended 2023 allowable biological catch (ABC) is 1,573,109 round
pounds (\(F_{ABC}\) = 0.063), a 9\% increase from the 2022 ABC (Table
@ref(tab:catchtab) and @ref(tab:brps)). After making decrements for
sablefish mortalities in other fisheries, \textbf{the 2023 NSEI
Subdistrict commercial sablefish fishery annual harvest objective (AHO)
is {1,395,686} round pounds} (Table @ref(tab:catchtab) and
@ref(tab:decrements)). There are 73 valid Commercial Fisheries Entry
Commission (CFEC) permits for 2023, which is the same number as 2022.
\textbf{The individual equal quota share (EQS) is {19,121 } round
pounds}, a {13\%} increase from the 2022 EQS of 16,899 round pounds
(Table @ref(tab:catchtab)).

Several advancements to the stock assessment and statistical
catch-at-age (SCAA) model were implemented for the 2023 NSEI sablefish
assessment that improved the model's ability to capture the dynamics of
the stock. These advancements are:

\begin{enumerate}
\def\labelenumi{\arabic{enumi}.}
\tightlist
\item
  Fishery CPUE was fully standardized to correct for variability in
  fishing methods and practices (i.e, hook size, fishing depth, length
  of sets, location, etc.) to better detect underlying trends that
  reflect the abundance of fish available to the fishery. This involved
  recalculating fishery CPUE from the re-entered logbook data that was
  completed in 2020.
\item
  Fishery selectivity in the SCAA was updated to the fixed values
  estimated in the federal sablefish fishery assessment (Goethel et
  al.~2022). Selectivity in the time period prior to implementation of
  the IFQ fishery in 1995 changed significantly from the last assessment
  such that the curve is substantially less steep indicating that fewer,
  smaller fish were being selected for in the pre-IFQ fishery than
  previously estimated. This has the effect of increasing the apparent
  size of the population during that time period and thus reducing stock
  status today. The model is now capable of estimating fishery
  selectivity within the model, but the estimated selectivity curves are
  suspect and further work needs to be done before this version of the
  model is implemented.
\item
  Survey selectivity was switched from the fixed values borrowed from
  the federal domestic longline survey values to being freely estimated
  in the SCAA model, thus being a more accurate reflection of the NSEI
  longline survey. This involved adding a second time block to reflect
  the switch from an unstandardized survey prior to 2000 and the fully
  standardized fishery that began in 2000.
\item
  The recruitment process is now modelled using random effects which
  allows for the estimation of variability, \(\sigma_R\). Prior to this
  assessment, \(\sigma_R\) had been fixed at the assumed federal
  assessment value of 1.2.\\
\item
  The data weighting of the model was changed to reflect best practices
  in SCAA modelling. This involved tuning the age and length
  compositional data to adjust the effective sample sizes using
  McAllister and Ianelli (1997) and removing the fixed weights that had
  been applied to the abundance indices (mark-recapture estimates,
  longline survey CPUE and longline fishery CPUE). The variance of the
  longline survey was changed from assumed values to the true estimates.
  The fishery CPUE and mark-recapture variance was kept at the inflated
  and fixed values to allow for the extra uncertainty in these indices
  owing to the unrecorded releases of fish that are permitted in the
  fishery and unquantified biases in the mark-recapture project.
\end{enumerate}

With these changes, the recommended 2023 ABC is 1,573,109 round lb
(\(F_{ABC}\) = 0.063), a 9\% increase from the 2022 ABC. The ABC was
calculated as an average of the base model and the new model (v23) to
balance the clear increase in biomass with the uncertainty about stock
status evident in comparing the base and v23 models. The increase in the
ABC is attributed to the continued growth and maturation of the strong
recruitment events since 2015, highlighted by recruitment in 2018 (the
2016 year class) which is the highest recruitment since 1979. The
dominant 2016 year class is now 50\% mature and will comprise 27\% of
the biomass. All three abundance estimates are elevated from recent
years with the highest abundance estimate on record from the
mark-recapture project, the third sustained year of high CPUE in the
longline survey and increasing CPUE in the longline fishery. However,
the lower stock status estimated in the new model results in less of an
increase than was present in using the 2022 model (the base model). The
recommended ABC is thus an average of the recommended ABC from the base
model used in past assessments and the new model v23 used in this
assessment. Fishery catch and ex-vessel value remain depressed from
historical levels, but have increased since 2022 as the 2013-2018 year
classes reach marketable sizes and are being landed and retained in the
fishery (Figure @ref(fig:catch)). Though recent high catch rates of
small sablefish across multiple geographic areas signal increasing
trends for sablefish stocks (Goethel et al.~2022), the department
maintains a precautionary approach to setting harvest limits. Estimates
from the 2022 stock assessment suggest sablefish spawning stock biomass
remains at suppressed levels compared to the 1980s and 1990s.

The ABC determination process uses a statistical catch-at-age model,
which was first implemented in 2020. The model reduces the reliance on
the annual mark-recapture project to estimate recruitment, abundance,
and spawning stock biomass of NSEI sablefish by integrating multiple
indices of abundance and biological data (e.g., catch, mark-recapture
abundance estimates, longline survey and fishery CPUE, longline survey
length and age compositions). As in previous years, maximum ABC is
defined by \(F_{50}\), the fishing mortality rate that reduces spawning
biomass to 50\% of equilibrium unfished levels.

The process leading to the determination of the ABC, AHO, and EQS
includes compiling fishery and survey data, running the stock
assessment, and accounting for additional sources of mortality through
decrements. Although the ABC is determined prior to the AHO and EQS,
this report is organized to make management-related information
accessible to stakeholders and improve documentation of the assessment
process by organizing this report into the following sections:

\begin{enumerate}
\def\labelenumi{\arabic{enumi}.}
\tightlist
\item
  2023 Sablefish Management Plan: details the decrements process leading
  to the AHO and EQS and effective regulations for the 2023 NSEI
  fishery.
\item
  2022 Sablefish Stock Assessment and 2023 ABC Determination: highlights
  stock assessment data inputs, methods, results, and subsequent
  analyses that informed the recommended ABC.
\end{enumerate}

\hypertarget{sablefish-management-plan}{%
\section{2023 SABLEFISH MANAGEMENT
PLAN}\label{sablefish-management-plan}}

\hypertarget{annual-harvest-objective-determination}{%
\subsection{ANNUAL HARVEST OBJECTIVE
DETERMINATION}\label{annual-harvest-objective-determination}}

The 2023 AHO was determined by making the following decrements from the
recommended ABC (1,573,109 1,573,109 round pounds, Tables @ref(tab:brps)
and @ref(tab:decrements)):

\begin{itemize}
\tightlist
\item
  estimated sablefish bycatch mortality in the commercial Pacific
  halibut fishery,
\item
  ADF\&G longline survey removals,
\item
  sport fishery guided and unguided harvest,
\item
  mortality from fishery deadloss, and
\item
  subsistence and personal use harvest.
\end{itemize}

\hypertarget{bycatch-mortality-in-the-halibut-fishery}{%
\subsubsection{Bycatch mortality in the halibut
fishery}\label{bycatch-mortality-in-the-halibut-fishery}}

Sablefish caught in NSEI during the Pacific halibut individual fishing
quota fishery prior to the sablefish fishery season opening (August 15)
must be released; however, because not all are expected to survive,
bycatch mortality is estimated. Prior to 2003, a 50\% bycatch morality
rate was applied as bycatch sablefish were permitted to be retained as
bait. In 2003, the Alaska Board of Fisheries disallowed retaining
bycatch sablefish for bait, and a 25\% bycatch mortality rate was
assumed for all sablefish caught and released due to the larger hook
size in the Pacific halibut fishery. Released sablefish bycatch is
calculated as the product of the 3-year average of the sablefish to
Pacific halibut ratio from the International Pacific Halibut Commission
(IPHC) annual survey and the 3-year average of the Pacific halibut catch
in areas greater than 99 fathoms in NSEI.

\hypertarget{adfg-longline-survey-removals}{%
\subsubsection{ADF\&G longline survey
removals}\label{adfg-longline-survey-removals}}

In 2022, no NSEI permit holders will participate in the NSEI longline
survey due to budgetary instability and deficit given the low prices of
sablefish in 2020 and 2021 (Tables 3 and 4). The survey removal
decrement was determined by averaging the survey total harvest from the
previous 3 years. Permit holders will likely resume survey participation
in 2023.

\hypertarget{sport-fish-harvest-guided-and-unguided}{%
\subsubsection{Sport fish harvest (guided and
unguided)}\label{sport-fish-harvest-guided-and-unguided}}

Sablefish sport fish preliminary harvest and release mortality from the
guided and unguided sectors are estimated utilizing charter logbooks and
the statewide harvest survey (Romberg et al.~2017). Estimates of
harvested and released fish are based on the total number of fish and
converted to weight using a 3-year average of fish sampled from the
guided and unguided sectors. A 10\% release mortality rate is applied to
the sport fishery; this was based on the 11.7\% estimated in Stachura et
al.~(2012) and modified to account for difference in gear type (rod and
reel versus longline) and handling time.

\hypertarget{mortality-from-fishery-deadloss}{%
\subsubsection{Mortality from fishery
deadloss}\label{mortality-from-fishery-deadloss}}

Deadloss mortality in the directed sablefish fishery was estimated by
applying the percentage of dead sablefish (i.e., recorded as predated by
sand fleas, sharks, hooking injury, or other cause of mortality) caught
on the NSEI longline survey using the recent 3-year average, 0.85\%
(2020--2022), to the NSEI sablefish commercial AHO.

\hypertarget{personal-use-and-subsistence-harvest}{%
\subsubsection{Personal use and subsistence
harvest}\label{personal-use-and-subsistence-harvest}}

A total of 772 personal use and subsistence sablefish permits were
issued in 2022. Annual subsistence and personal use harvest of sablefish
is estimated from these permits by adding the total number of retained
sablefish reported to the proportion of released sablefish reported
after applying a 16\% discard mortality rate to released sablefish
(Gilroy and Stewart 2013). The Pacific halibut fishery is assumed a
reasonable proxy for sablefish because the fisheries utilize similar
gear and frequently the same vessels and crew participate in both
fisheries. Moreover, both species are considered hardy and do not
experience barotrauma. The 2022 longline survey average weight (5.1 lb)
was applied to this harvest to obtain a decrement total. In 2015,
personal use harvest was limited to an annual limit of 50 fish per
household. Since 2018, participants of the personal use fishery have
been allowed to use pot gear with no more than 2 pots per permit and a
maximum of 8 pots per vessel when 4 or more permit holders are on board
the same vessel. Use of pot gear has continued to increase with 68\% of
permit holders fishing pots.

\hypertarget{regulations}{%
\subsection{REGULATIONS}\label{regulations}}

\hypertarget{board-of-fisheries-decisions}{%
\subsubsection{2022 Board of Fisheries
Decisions}\label{board-of-fisheries-decisions}}

In March 2022, the Alaska Board of Fisheries adopted new regulations
that will be enacted prior to or during the fishing season for the NSEI
sablefish commercial fishery. An advisory announcement will be issued at
a later date with more information. These new regulations include:

\begin{itemize}
\tightlist
\item
  Full retention requirements and landing requirements using
  hook-and-line and pot gear for all species of rockfish including
  thornyhead rockfish.
\item
  Allowing pot gear as a legal gear type in addition to longline gear
  for the C61A permits, which is contingent upon the approval process
  through CFEC.
\item
  Pots must have at least two circular escape rings, with a minimum
  inside diameter of three and three-fourths inches, installed on
  opposing vertical or sloping walls of the pot.
\end{itemize}

\hypertarget{registration-and-logbook-requirements}{%
\subsubsection{Registration and logbook
requirements}\label{registration-and-logbook-requirements}}

Fishermen must register prior to fishing {[}5 AAC 28.106 (b){]} and keep
a logbook during the fishery. Completed logbook pages must be attached
to the ADF\&G copy of the fish ticket at the time of delivery.
Confidential envelopes for logbook pages may be requested when
registering.

Permit holders will receive a personal quota share (PQS) tracking form
at the time of registration. This form is used to record the total round
weight landed for each delivery. Each permit holder must, upon request,
provide the buyer with the total round weight of sablefish the permit
holder has landed to date. The department requests that a copy of the
completed PQS tracking form is included with the final fish ticket of
the season for that permit.

Logbooks must include, by set, the date and time gear is set and
retrieved, specific location of harvest by latitude and longitude for
start and ending positions, hook spacing, amount of gear (number of
hooks and skates) used, depth of set, estimated number or weight of the
target species, and the estimated number or weight of bycatch by
species. Permit holders must indicate for each set if the target species
was sablefish or Pacific halibut and if there was any gear lost. A
permit holder must retain all visibly injured or dead sablefish.
Sablefish that are not visibly injured or dead may be released unharmed,
and the permit holder must record in the logbook, by set, the number of
live sablefish released {[}5 AAC 28.170(f){]}. Permit holders must
record release reason (e.g., fish are small) and whether their personal
quota share has been met.

\hypertarget{tagged-sablefish}{%
\subsubsection{Tagged sablefish}\label{tagged-sablefish}}

Fishermen are requested to watch for tagged sablefish, record tag
number(s), and attach tags directly in the logbook with the
corresponding set information. All tags returned will receive a reward.
Tag rewards include a t-shirt and entry into an annual drawing for one
\$1,000, two \$500, and four \$250 cash rewards. To qualify for entry in
the annual drawing, ADF\&G requires the following information: the tag,
set location (latitude and longitude), date of capture of the fish, and
the name and address of the person recovering the tag.

\hypertarget{sablefish-possession-and-landing-requirements}{%
\subsubsection{Sablefish possession and landing
requirements}\label{sablefish-possession-and-landing-requirements}}

In the NSEI Subdistrict, the holder of a CFEC permit for sablefish may
not retain more sablefish from the directed fishery than the annual
amount of sablefish EQS specified by the department {[}5 AAC 28.170
(f){]}. However, if a permit holder's harvest exceeds the EQS for that
year, by not more than 5\%, ADF\&G shall reduce the permit holder's EQS
for the following year by the amount of the overage. If a permit
holder's harvest exceeds the permit holder's EQS by more than 5\%, the
proceeds from the sale of the overage in excess of 5\% shall be
surrendered to the state and the permit holder may be prosecuted under
AS 16.05.723 {[}5 AAC 28.170 (j){]}. If a permit holder's harvest is
less than the permit holder's EQS established for the year, ADF\&G shall
increase the permit holder's PQS only for the following year by the
amount of the underage that does not exceed 5\% of the EQS {[}5 AAC
28.170 (k){]}. For the 2023 fishing season, 5\% of the annual EQS is
956.05 round pounds.

\hypertarget{fish-tocket-requirements}{%
\subsubsection{Fish tocket
requirements}\label{fish-tocket-requirements}}

Landed weights must be recorded on a fish ticket at the time of
delivery. If a fisherman delivers fish in the round, the total round
weight delivered must be recorded on the fish ticket. If a fisherman
delivers dressed fish, the fish ticket must include the total landed
dressed weight as well as the round weight equivalent, determined by
using the standard 0.63 recovery rate. There is a 2\% allowance for ice
and slime when unrinsed whole iced sablefish are weighed. A fish ticket
must be completed prior to the resumption of fishing and each permit
holder must retain, on board their vessel, copies of all NSEI sablefish
tickets from the current season and their updated PQS tracking form.
When delivering fish out of state, a completed fish ticket must be
submitted to ADF\&G prior to transporting fish out of Alaska.

\hypertarget{bycatch-allowances-for-other-species}{%
\subsubsection{Bycatch allowances for other
species}\label{bycatch-allowances-for-other-species}}

Full retention and reporting of rockfish \emph{Sebastes}, excluding
thornyhead rockfish \emph{Sebastolobus}, is required for internal waters
(5 AAC 28.171). The full retention regulation does not apply to
thornyhead rockfish at the time of publication but will when the new
regulations become effective in 2022. The allowable bycatch that may be
legally landed and sold on an NSEI sablefish permit is based on round
weight of sablefish and bycatch species or species group on board the
vessel:

\begin{itemize}
\tightlist
\item
  All rockfish, including thornyheads: 15\% in aggregate, of which 1\%
  may be demersal shelf rockfish (DSR), which includes yelloweye,
  quillback, canary, tiger, copper, China, and rosethorn rockfish
\item
  Lingcod: 0\%
\item
  Pacific cod: 20\%
\item
  Spiny dogfish: 35\%
\item
  Other groundfish: 20\%
\end{itemize}

All rockfish retained in excess of allowable bycatch limits shall be
reported as bycatch overage on an ADF\&G fish ticket. All proceeds from
the sale of excess rockfish bycatch shall be surrendered to the state.
Excess rockfish retained due to full retention requirements may be
retained for personal use; however, the pounds must be documented as
overage on the fish ticket.

A CFEC permit holder fishing for groundfish must retain all Pacific cod
when the directed fishery for Pacific cod is open and up to the maximum
retainable bycatch amount (20\%) of Pacific cod when a directed fishery
for Pacific cod is closed {[}5 AAC 28.070 (e){]}. Pacific cod taken in
excess of the bycatch limit in areas open to directed fishing for
Pacific cod may be landed on a CFEC miscellaneous saltwater finfish
permit designated for the gear that was used. Fishermen with halibut
Individual Fishing Quota (IFQ) in regulatory area 2C and a CFEC halibut
permit card must retain all halibut over 32 inches in length, up to the
amount of their IFQ.

\hypertarget{sablefish-live-market}{%
\subsubsection{Sablefish live market}\label{sablefish-live-market}}

The holder of a CFEC or interim use permit for sablefish may possess
live sablefish for delivery as live product except that, upon request of
a local representative of the department or law enforcement, a permit
holder must present sablefish for inspection and allow biological
samples to be taken {[}5 AAC 28.170 (l){]}.

\hypertarget{prohibitions}{%
\subsubsection{Prohibitions}\label{prohibitions}}

The operator of a fishing vessel may not take sablefish in the NSEI area
with sablefish from another area on board. Also, the operator of a
vessel taking sablefish in the NSEI area shall unload those sablefish
before taking sablefish in another area {[}5 AAC 28.170(a) and (b){]}.

A vessel, or person onboard a vessel, from which commercial,
subsistence, or personal use longline fishing gear was used to take fish
in the NSEI or SSEI Subdistricts during the 72-hour period immediately
before the start of the commercial sablefish fishery in that
subdistrict, or from which that gear will be used during the 24-hour
period immediately after the closure of the commercial sablefish fishery
in that subdistrict, may not participate in the taking of sablefish in
that subdistrict during that open sablefish fishing period. A vessel, or
a person onboard a vessel, who has harvested and sold their personal
quota share before the final day of the sablefish season in that
subdistrict is exempt from the prohibition on fishing longline gear
during the 24-hour period immediately following the closure of the
sablefish fishery in that subdistrict. In addition, a vessel or a person
on board a vessel commercial fishing for sablefish in the NSEI
Subdistrict may not operate subsistence or personal use longline gear
for groundfish from that vessel until all sablefish harvested in the
commercial fishery are offloaded from the vessel.

\hypertarget{sablefish-stock-assessment-and-2023-recommended-abc-determination}{%
\section{2022 SABLEFISH STOCK ASSESSMENT AND 2023 RECOMMENDED ABC
DETERMINATION}\label{sablefish-stock-assessment-and-2023-recommended-abc-determination}}

Sablefish are a highly migratory, long-lived species broadly distributed
in the North Pacific Ocean. Although research to date suggests that
sablefish comprise a single, panmictic population, they are managed as
separate stocks in Alaska state and federal waters, British Columbia,
and in state and federal waters off the U.S. west coast. After three
decades of declining or suppressed spawning stock biomass in the North
Pacific, persistent high catch rates of small sablefish in recent years
across multiple surveys and fisheries signal strong recruitment and
increasing trends for the stock (Goethel et al.~2021).

Despite these positive population trends, we continue to recommend a
precautionary approach to setting harvest limits. The target fishing
mortality rate of \(F_{50}\), that defines maximum ABC is based on
female spawning stock biomass and does not take into account the
relative economic value of sablefish. Because sablefish begin
contributing to the spawning biomass as young as age-3, ABCs can
increase quickly even if average fish size is small. These small
sablefish are worth significantly less per pound, making them subject to
high release rates in NSEI where fishery releases are legal. Taken
together, steep annual increases in ABCs in response to large
recruitment events can result in low fishery value, and the unobserved
fishery releases introduce an uncertain source of mortality into the
stock assessment. As the 2013-2018 year classes mature these strong
recruitment events are beginning to translate into higher catches and
ex-vessel value evident in 2022 (Figure @ref(fig:catch)). CPUE in the
fishery has increased as more of these fish are landed, which tracks
with increased CPUE in the longline survey and the high estimates of
abundance from the mark-recapture project. As fish from these strong
year classes grow they are more likely to be retained and sold.
Similarly, as these fish mature they are increasing the size of the
spawning biomass.

In response to concerns about release practices, we introduced a ``max
15\% change'' management procedure in 2020 that constrains the
recommended ABC to a 15\% annual maximum change. This management
procedure was well-received during two stakeholder and industry meetings
in April 2020 and 2021 and appears to be supported by the fleet. The
``max 15\% change'' management procedure has been shown to increase
fishery stability, maximize catch, and successfully achieve biological
goals in long-term simulations conducted by IPHC
(\url{https://www.iphc.int/uploads/pdf/srb/srb014/ppt/iphc-2019-srb014-08-p.pdf}).
The current NSEI harvest policy continues to define maximum permissible
ABCs at \(F_{50}\), and recommended ABCs will be constrained to a
maximum 15\% change between years.

In 2020, we implemented an integrated statistical catch-at-age (SCAA)
model for the NSEI stock assessment, which had been in development for
several years (Sullivan et al.~2020). The SCAA model is structured
similarly to the federal sablefish model (Goethel et al.~2022) and
allows for the estimation of recruitment, spawning stock biomass, and
abundance. This model was used again in 2023 with several modifications
that loosened reliance on fixed values derived from the federal
assessment and makes the model more responsive to NSEI specific data.

The SCAA model results in a maximum permissible ABC of 1,573,109 round
lb at a target fully selected fishing mortality of \(F_{50}\) (Table
@ref(tab:brps)). This is a 129,795 lb increase (9\%) from the 2022 ABC
of 1,443,314 round lb. Under the max 15\% change management procedure,
the recommended 2023 ABC remains the same as the maximum permissible
ABC. To account for legal releases of small sablefish in NSEI, fixed
retention probabilities and an assumed discard mortality of 16\% were
incorporated directly into the SCAA model following Sullivan et
al.~(2019). The mortality from fishery releases under \(F_{50}\) is
estimated to be 69,522 lb (79,711 in the base model) and is incorporated
directly into the max ABC calculation. See section titled ``ABC
Recommendations'' for more information.

The following are notable results from the SCAA model and reflect
potential conservation or assessment concerns for this stock:

\begin{enumerate}
\def\labelenumi{\arabic{enumi}.}
\item
  {Slinky pots:} This was the first year where slinky pots were allowed
  in the NSEI sablefish fishery. Participation was limited in 2022 and
  did not impact this assessment. A large increase in pot usage is
  anticipated in 2023 and will likely affect the next assessment in
  2024.
\item
  Stock status (i.e, the proportional relationship of the stock relative
  to it's virgin state) remains uncertain and sensitive to data
  weighting methodology and fishery selectivity values that remain fixed
  to values from the federal assessment. The department manages the NSEI
  fishery for \(F_{50}\) (the fishing mortality that results in a SPR
  (spawner per recruit) of 50\%) and changes in model structure and
  assumptions results in changes to where the population is relative to
  this target (Table @ref(tab:modcomp) and Figure @ref(fig:Fsprcomp)).
  Updates to the model instituted in this assessment decreased the
  reliance on subjective weighting of the data sources but retains a
  degree of subjectivity in the amount of variance ascribed to the three
  indices of abundance. The model relies on variance terms for the
  mark-recapture abundance estimate and fishery CPUE that are inflated
  above those calculated from the data and changes to those terms
  results in different conclusions about where the population is
  relative to the management target. Furthermore, updating fishery
  selectivity to the most recent estimates available from the federal
  model affects estimates of stock status and is thus a source of
  concern given likely differences in fishery selectivity between the
  federal and NSEI fisheries. Although this assessment demonstrates a
  9\% increase in the ABC from last year, permutations to the variance
  terms associated with those indices can swing that increase to as low
  as 3\% (Table @ref(tab:modcomp)). The trend in the stock is
  unambiguous, but managers should be wary of the uncertainty inherent
  in the current operating model. As the data weighting for this model
  continues to evolve to be in line with best practices a goal remains
  to remove subjective assignment of variances or weighting and allow
  the model to estimate variance beyond that calculated from the data.
  Initial steps were taken to address these concerns in 2023 but require
  more work before they are adopted into the operating model.
\item
  Fit of the model to the abundance indices remains poor and reliant on
  the inflated variance terms assigned to fishery CPUE and
  mark-recapture estimates of abundance. In particular, the abundance
  estimates derived from the mark-recapture assumption that have
  underpinned the NSEI sablefish assessment since 2005 and provides
  scale to the population now appears to underestimate abundance
  relative to the model estimates. There is tension between the other
  data sources (age and length compositions) that forces the
  aforementioned data weighting to keep the model tethered to those
  abundance estimates. A thorough review of the mark-recapture
  experiment to identify and correct biases in the estimate remains a
  priority for this project. Bias correction may result in better fit to
  the model both by correcting estimates and modifying the modeling
  prior (penalized likelihood) describing the relationship between
  actual abundance and the mark-recapture estimate (currently assumed to
  be a 1:1 ratio).
\item
  Fixing fishery selectivity to values estimated in the federal
  assessment remains a principal weakness in this model and assessment.
  Efforts were made in this year's assessment to estimate fishery
  selectivity in the model, but thus far the model has failed to
  converge and it was necessary to leave fishery selectivity fixed for
  this assessment. The selectivity curve for the derby (pre-IFQ) fishery
  changed substantially since the last assessment owing to a general
  lack of data for the pre-IFQ fishery in the federal assessment. The
  model and stock-status estimates remain sensitive to these selectivity
  values and developing the model to estimate fishery selectivity in the
  NSEI fishery remains a high priority going forward.
\item
  The fit of the model to the age data has improved relative to past
  assessments and is the result of the model tuning that resulted in
  higher estimates of effective sample size than those used in past
  assessments (a conservative estimate derived from the square root of
  the raw sample size). The fit is still not satisfactory and likely is
  the result of fixed selectivity values for the fishery. Better
  estimated selectivity curves remain a priority for addressing the fit
  of age data.
\item
  Similarly, while the fit to length data is also improved owing to that
  model tuning methods now used, it is still far from desirable with a
  consistent pattern in the residuals whereby mid-size fish are
  underestimated and larger fish are overestimated in the model. In
  conjunction with the retrospective results this suggests that the
  model may be underestimating large recruitment events. Better
  estimation of selectivity in both the fishery and the survey will be
  necessary to improve the fit of length data.
\item
  Recruitment of the 2013-2018 year classes was substantial and above
  the long term average. These strong year classes are driving the
  increase in biomass that has occurred over the last several years.
  These recruitment events are in line with what is seen in the federal
  assessment, although the increase evident in the NSEI population is
  not as steep as that seen in the federal assessment. These fish are
  still not fully mature or fully grown and thus biomass is likely to
  continue increasing over the next several yers as these fish grow and
  mature into the population. However, they are likely still less than
  optimal size from a price standpoint and may still be subject to high
  release rates.
\item
  Retrospectives patterns in the model are satisfactory. The model
  demonstrates a slightly positive bias in spawning biomass of 5\%
  indicating that the model tends to overestimate spawning biomass. The
  bias in recruitment is also low on average, although individual years
  can be quite biased (up to 200\%). In general, the model overestimates
  recruitment during low recruitment periods and underestimates
  recruitment during periods of high recruitment. Given the strong
  evidence that the population has experienced a recruitment boom over
  the last several years it is likely that the size of those year
  classes is somewhat underestimated and the population will see
  continued growth for several more years.
\end{enumerate}

\hypertarget{changes-to-the-nsei-assessment-relative-to-2021}{%
\subsection{CHANGES TO THE NSEI ASSESSMENT RELATIVE TO
2021}\label{changes-to-the-nsei-assessment-relative-to-2021}}

Updates to the stock assessment are listed here:

\begin{enumerate}
\def\labelenumi{\arabic{enumi}.}
\item
  Fishery CPUE was fully standardized to control for variability in
  fishing methods and practices over time (i.e, hook size, fishing
  depth, length of sets, location, etc.) to better detect underlying
  trends that reflect the abundance of fish available to the fishery.
  This involved recalculating fishery CPUE from the re-entered logbook
  data that was completed in 2020. In 2020, the ADF\&G Southeast
  Groundfish Project biologists invested considerable staff time and
  resources into standardizing the full time series of available logbook
  data, which should improve the long-term quality and interpretation of
  this index. In particular, consistent methods for identifying target
  species by trip and set efforts were developed, which was previously
  conducted manually. {METHODS FROM RHEA!!!}
\item
  Fishery selectivity in the SCAA was updated to the fixed values
  estimated in the federal sablefish fishery (Goethel et al.~2022).
\item
  Survey selectivity was switched from being fixed to the values
  estimated in the federal domestic longline survey to being freely
  estimated in the SCAA model, thus being a more accurate reflection of
  the NSEI longline survey. This involved modelling selectivity in two
  time blocks reflecting the survey before and after it became fully
  standardized in 2000.
\item
  The recruitment process is now modelled using random effects which
  allows for the estimation of variability, \(\sigma_R\). Prior to this
  assessment, \(\sigma_R\) had been fixed at the assumed federal
  assessment value of 1.2.
\item
  The data weighting of the model was changed to reflect best practices
  in SCAA modelling. This involved tuning the age and length
  compositional data to adjust the effective sample sizes using
  McAllister and Ianelli (1997) methodology and removing the fixed
  weights that had been applied to the abundance indices (mark-recapture
  estimates, longline survey CPUE and longline fishery CPUE). The
  variance of the longline survey was changed from assumed values to the
  true estimates of variance. The fishery CPUE and mark-recapture
  variances were kept at the inflated and fixed values to allow for the
  extra uncertainty in these indices owing to the unrecorded releases of
  fish that are permitted in the fishery.
\end{enumerate}

We made no additional changes to the SCAA model structure or
assumptions, estimation of biological reference points, or population
dynamics equations. We used status quo methods to update estimates of
weight-at-age, maturity-at-age, catch, survey CPUE, mark-recapture
abundance, and age/length compositions. For detailed technical
information on the SCAA model and data preparation, please see Sullivan
et al.~(2020) or visit the GitHub repository for this project:
\url{https://github.com/commfish/seak_sablefish}.

\hypertarget{model-structure}{%
\subsection{MODEL STRUCTURE}\label{model-structure}}

The integrated statistical catch-at-age (SCAA) model presented here was
coded in TMB, an R library that leverages C/C++ functionality to
calculate first and second order derivatives and was inspired by a
similar C/C++ templating software ADMB (Kristensen et al.~2016; Fournier
et al.~2012). The TMB code replicates or makes refinements to methods
used in a previous ADMB based, age-structured model for the NSEI
sablefish stock (Mueter 2010) that was based on code from an older
federal assessment of sablefish that has also been adapted for several
Alaska rockfish stocks (Kimura 1990; Sigler 1999). The model can be run
as either a single-sex or sex-structured model; however, data inputs are
only shown for the sex-structured model. Variable definitions for all
equations used in the statistical catch-at-age model can be found in
Table @ref(tab:vardefns). Uncertainty in parameters are currently
estimated using a maximum likelihood approach.

\hypertarget{data-inputs}{%
\subsection{DATA INPUTS}\label{data-inputs}}

The data used as inputs to the SCAA model biological data, catch,
abundance, and composition (Figure 4) are found here:
\url{https://github.com/commfish/seak_sablefish/tree/master/data/}
tmb\_inputs).

\hypertarget{weight-at-age}{%
\subsubsection{Weight-at-age}\label{weight-at-age}}

Data from the 2002--2022 longline fishery and 1997--2022 ADF\&G longline
surveys were used to obtain fishery and survey weight-at-age used in the
SCAA model. A weight-based von Bertalanffy growth model was fit to
weight-at-age data:

\[
\begin{equation}
\ln(w_a)=\ln W_{\infty}+\beta \cdot \ln (1-\exp (a-t_0)) \\
(\#eq:weightvonb)
\end{equation}
\]

where \(w_a\) is weight at a given age (lb), \(W_{\infty}\) is the mean
asymptotic weight (lb), \(\beta\) is the power in the allometric
equation, \emph{k} relates to the rate at which \(W_{\infty}\) is
reached, and \(t_0\) is the theoretical age at weight zero (years).
Residuals \(\epsilon\) were assumed lognormally distributed to account
for increasing variability by age, and the variance of these residuals
(\(\sigma_R\)) was estimated. Models were fit separately for each sex
and data source using maximum likelihood and the mle() function in R.

The federal assessment uses survey weight-at-age exclusively to fit to
catch and effort indices (Hanselman et al.~2018). However, because
discarding is permitted in the NSEI fishery, there are large differences
in survey and fishery weight-at-age, especially at younger ages (Figure
@ref(fig:compcomps)A). Consequently, fishery weight-at-age was fit to
landed catch biomass, whereas survey weight-at-age was used to estimate
exploitable biomass, spawning biomass, and other quantities of interest
in the model.

\hypertarget{maturity-at-age}{%
\subsubsection{Maturity-at-age}\label{maturity-at-age}}

Maturity data from the 1997--2022 ADF\&G longline surveys were used to
fit a maturity ogive for female sablefish using logistic regression and
the glm() function in R. Maturity-at-length data for this time period
were more abundant than maturity-at-age data and appeared to provide the
best estimates of maturity; therefore, maturity curves were fit using
maturity-at-length data.

Predicted maturity-at-length was transformed to maturity-at-age using
fitted values from a length-based von Bertalanffy growth curve fit to
survey data. The length at 50\% maturity is 61.2 cm; the \(k_{mat}\)
(the slope at the length at 50\% maturity) is 0.38; and the age at 50\%
maturity is 5.9 years (Figure @ref(fig:bioinputs)). Predicted
proportions maturity-at-age were used as inputs to the SCAA model and in
the calculation of spawning stock biomass.

Annual fits of maturity, though not explicitly used in the SCAA model,
can provide insight into changes in the population or cohort-specific
dynamics. Of note, the fit to maturity data in the last four years
suggest that fish matured at younger ages and smaller sizes compared to
previous years (Figure @ref(fig:deltamat)). It is possible that earlier
maturation can be linked to warm environmental conditions in the North
Pacific since 2014, or to density-dependent effects driven by the large
recruitment events in recent years. Trends in maturity and growth should
be monitored in future assessments.

\hypertarget{catch}{%
\subsubsection{Catch}\label{catch}}

Catch data from 1975 to 2022 include harvest in the directed sablefish
longline and pot fishery, ADF\&G longline survey removals, and sablefish
retained in other fisheries like the individual fishing quota halibut
longline fishery (Figure @ref(fig:catch) and @ref(fig:abdind)A). Catch
estimates from 1975 to 1984 were obtained from Carlile et al.~(2002) and
1985--present catch was obtained from fish tickets. Catch was estimated
in the SCAA model assuming a lognormal distribution with a fixed log
standard deviation of 0.05. Changes in the management structure during
this period included a move to limited entry in 1985 and the EQS program
in 1994 (Olson et al.~2017). Additional sources of mortality that are
not currently included in this model include sport, subsistence and
personal use harvest, estimated bycatch mortality in the halibut
fishery, and estimated deadloss including mortality from sand fleas,
sharks, and whales. Currently these additional sources of mortality are
accounted for in the decrements process (see the section titled ``Annual
Harvest Objective Determination'' for more information).

\hypertarget{fishery-cpue}{%
\subsubsection{Fishery CPUE}\label{fishery-cpue}}

Fishery CPUE, defined as retained lb per hook, was used as an index of
abundance from 1980 to 2022 (Figure @ref(fig:abdind)B). Fishery CPUE was
estimated in the SCAA model assuming a lognormal distribution with a
fixed log standard deviation of 0.1 for the historical data from
dockside interviews (1980--1996; Carlile et al.~2002) and 0.08 for the
contemporary logbook data (1997--present).

{In 2020, ADF\&G reviewed and re-entered logbook data to standardize how
trip and set targets were identified. In past years this was done ad-hoc
on an annual basis and methods were not documented leading to confusion
with the retirement and turnover of staff. This project established
guidelines for identifying trip and set target based on yadda, yadda,
yadda. Trips that had CONDITION were used to identify trip targets while
sets that had CONDITION were used to identify set targets. Only sets and
trips targetting sablefish were used to calculate fishery CPUE values
for use in the assessment. }

Fishery CPUE since 1997 was fully standardized in this year's assessment
to account for shift in fishing practices and vessel participation over
time. Standardization accounts for variability in hook size, hook
spacing, fishing depth, soak time, statistical area (fishing location),
fishing vessel (as a random effect), julian day, and set length. CPUE
was estimated as the predicted value from generalized additive models
(GAM) fitting CPUE to these variables using the mgcv package in R and
the gamma smoothing feature. Standardization resulted in slight changes
in the overall time series from past assessments, but the standardized
values do a better job of catpuring the increase in biomass that has
occured in recent years (Figure @ref(fig:fcpuestand)). Standardized
fishery CPUE in 2022 was at it's highest value since 2000 (Figure
@ref(fig:abdind)B), although it remains below the high catch rates seen
in the 1980s and early 1990s (Figure @ref(fig:abdind)B and Figure
@ref(fig:fcpuestand)).

Because discarding sablefish is legal in the NSEI fishery, estimating
fishery selectivity within the model is not currently possible. To
address this issue, the federal selectivity curve is used in the model,
which is estimated assuming 100\% mandatory retention. A sex- and
age-specific retention probability, coupled with a fixed discard
mortality rate, are used to estimate mortality from fishery releases.
Future research will be aimed at better understanding discarding
behavior in the NSEI fishery as it relates to economic and biological
factors.

\hypertarget{survey-cpue}{%
\subsubsection{Survey CPUE}\label{survey-cpue}}

Longline survey CPUE in numbers per hook was used as an index of
abundance from 1997 to 2022 (Figure @ref(fig:abdind)C). This index was
assumed to be log-normally distributed, with a fixed log standard
deviation derived from the data. The 1988--1996 longline surveys used a
shorter soak time of 1 hr instead of the current 3--11 hr (Carlile et
al.~2002; Dressel 2009). These data were omitted because the 1 hr soak
time was likely too short to provide an accurate measure of relative
abundance (Sigler 1993).

Survey CPUE has remained substantially above the long term mean since
2020 with minimal variation over the last three years (Figure
@ref(fig:abdind)C).

\hypertarget{mark-recapture-abundance}{%
\subsubsection{Mark-recapture
abundance}\label{mark-recapture-abundance}}

Currently, ADF\&G conducts an annual mark--recapture survey that serves
as the basis for stock assessment and management (Green et al.~2016;
Stahl and Holum 2010). Fish are tagged during a pot survey in May and
June, with recaptures occurring in the ADF\&G longline survey in late
July or early August and the longline fishery from August through
November (Beder and Stahl 2016).

The mark--recapture abundance estimates provide an index of exploitable
abundance for years when a marking survey occurred (2003--2010, 2012,
2013, 2015, 2017--2020; 2022; Figure @ref(fig:abdind)D). This index was
assumed to be lognormally distributed with a fixed log standard
deviation of 0.05. The mark--recapture abundance index increased from
3.01 to 3.14 million fish (+4.3\%) between 2020 and 2022 and is the
highest estimate since 2005 (Figure @ref(fig:abdind)D).

The 2022 marking survey released 8,654 tagged fish (Table @ref(tab:mr)).
Following methods in past assessments, we accounted for tags recovered
outside of the NSEI or period of recapture, natural and fishing
mortality, and differences in the size of fish captured in the pot
survey and the longline fishery (Appendix A in Sullivan et al.~2019). A
summary of data used in the mark--recapture models is in Table
@ref(tab:mr).

Mark--recapture abundance estimates were obtained using a
time-stratified Petersen mark--recapture model implemented in the
Bayesian software JAGS 4.3.0 (Depaoli et al.~2016). For any given time
period 𝑖, the number of tagged fish in Chatham Strait (\emph{K}) and
subsequent abundance (\emph{N}) were modeled as:

\[
\begin{equation}
K_{i} = \left\{ \begin{array}{ll}
(K_0 - D_0)*exp(-M*t_i) &i = 1\\
(K_{i-1} - k_{i-1} - D_{i-1})*exp(-M*t_i) & i > 1
\end{array}\right.
(\#eq:mr1)
\end{equation}
\]

and

\[
\begin{equation}
N_{i} = \left\{ \begin{array}{ll}
N_i *exp(-M*t_i) &i = 1\\
(N_{i-1} - C_{i-1})*exp(-M*t_i) & i > 1
\end{array}\right.
(\#eq:mr2)
\end{equation}
\] where \(K_0\) is number of tags released in the ADF\&G pot survey,
\(D_0\) is the number of tagged fish that are not available to either
the ADF\&G longline survey or to the fishery (tags recovered in halibut
fishery or outside of Chatham Strait), \emph{𝑀} is assumed natural
mortality of 0.10 (Johnson and Quinn 1988), \emph{k} is the number of
marked fish recovered, and \emph{C} is the total catch or number of
sablefish removed. \(N_1\) was assumed to follow a normal distribution
with an uninformed prior (precision = \(1×10^{-12}\)) centered on past
assessments' forecast of abundance.

The probability that a sablefish caught in a given time period is marked
\(p_i\) is informed by the ratio of marks in the population to the total
population at that time \(K_i/N_i\). Each \(p_i\) is assumed to follow
beta prior distribution \(p_i\) = \(\beta\)(\(\alpha\), \(\beta\)),
where \(\alpha\) = (\(K_i/N_i\) ) ∗ \emph{x}, \(\beta\) = (\emph{1 −
\(K_i/N_i\) })/\emph{x}, and a large \emph{x} indicates confidence in
\(K_i/N_i\). Because \(N_i\) was previously assumed to follow vague
normal prior, \(p_i\) was assigned an informed prior by setting \emph{x}
equal to 10,000.

In each time period, the likelihood of recapturing \emph{k} marked
sablefish given \emph{n} sampled fish follows a binomial distribution,
where

\[
\begin{equation}
Pr(k|n,p) = {n \choose k}p^k(1-p)^{n-k}
(\#eq:mr3)
\end{equation}
\] Additional information on mark--recapture modeling, alternative
models considered, and model selection methodology is detailed in
Appendix A of Sullivan et al.~(2019).

The mark-recapture experiment likely overestimates precision and is
biased to some degree given that there are currently no diagnostics that
examine differences in capture probability based on fish size and/ or
location. Furthermore, the project relies on marked fish being returned
by fishermen and the accounting done at processing plants by ADF\&G
staff and tag returns from industry seldom agree. A thorough
re-evaluation of this project remains a priority both to detect and
potentially correct biases in the estimates and produce more accurate
estimates of uncertainty in the estimate.

\hypertarget{age-compositions}{%
\subsubsection{Age compositions}\label{age-compositions}}

Fishery age compositions from the 2002--2022 longline fishery and survey
age compositions from the 1997--2022 longline surveys (Figure
@ref(fig:agecomps)) were included in the model. The plus group age was
updated from 42 to 31 in 2020 to maintain consistency with the federal
assessment. Sample sizes were deemed insufficient to fit age
compositions by sex, so age data have been aggregated for both the
survey and fishery. The McAllister and Ianelli (1997) method of tuning
composition data by iteratively reweighting the sample size has been
applied to the SCAA model and were implemented in this assessment.

Currently no NSEI-specific ageing error matrix exists. Until this has
been fully developed and reviewed, the federal sablefish ageing error
matrix has been made available to the State of Alaska (D. Hanselman,
Fisheries Research Biologist, NOAA, Juneau, personal communication,
April 2019; Hanselman et al.~2018; Heifetz et al.~1999). The ageing
error matrix (\(\Omega_{a',a}\)) is the proportion observed at age
\emph{a} given the true age \({a}^{'}\). Ageing error matrices are
critical for correcting observed age compositions and estimating
recruitment (Fournier and Archibald 1982). Future research should
include the development of an ageing error matrix for NSEI in
conjunction with the ADF\&G Age Determination Unit.

\hypertarget{length-compositions}{%
\subsubsection{Length compositions}\label{length-compositions}}

Sex-structured length data from the 2002--2022 longline fishery and
1997--2022 ADF\&G longline surveys (Figure @ref(fig:lencomp)) were
summarized using the federal conventions for length compositions
(Hanselman et al.~2018). The federal assessment uses 2 cm length bins
ranging from 41 to 99 cm. Fish less than 41 cm (\emph{\(l_0\)}) were
omitted from the analysis, and fish greater than 99 cm were aggregated
into the 99 cm length bin (\emph{l+}). Effective sample sizes were
estimated using the McAllister and Ianelli (1997) method of tuning
composition data by iteratively reweighting the sample size.

Length distributions in the fishery have dramatically different patterns
than the survey (Figures @ref(fig:lencomp) and @ref(fig:compcomps)),
with few lengths in the fishery less than 60 cm. Full retention is not a
requirement in state waters and the length differences between the
survey and fishery are attributed to fishery releases of small fish.
Because of the bias introduced by allowing fish to be released in the
fishery, fishery age and length compositions tend to be poorly fit by
the model.

Finally, the selective harvest of larger-bodied fish results in large
differences between survey and fishery size-at-age. Until an age-length
key is developed for NSEI, the federal age-length keys
(\(\Lambda_{a,l,k}\)) will be used to fit both survey and fishery length
compositions (D. Hanselman, Fisheries Research Biologist, NOAA, Juneau,
personal communication, April 2019; Hanselman et al.~2018; Echave et
al.~2012). Ultimately, separate age-length keys should be developed for
each data source to account for the differences in survey and fishery
size-at-age.

\hypertarget{retention-probability}{%
\subsubsection{Retention probability}\label{retention-probability}}

The release of healthy (i.e., not dead, sand flea bitten, etc.)
sablefish is allowed in state waters. To model the discarding behavior
in the NSEI fishery, processor grade and price per pound data were used
to inform retention probabilities-at-size (Figure @ref(fig:retention)).
Based on conversations with groundfish port sampling staff and
fishermen, the lower bound of the Grade 2/3 (3.1 round lb) was assigned
a 10\% retention probability, the lower bound of the Grade 3/4 (4.9
round lb) was assigned a 50\% retention probability, and everything
greater than 8 round lb was assigned a 100\% retention probability (A.
Olson, Groundfish project leader, ADF\&G, personal communication, July
2018). Remaining retention probabilities were interpolated between these
fixed values. Weight-based retention probabilities were translated to
sex and age using the longline survey sex- and weight-based von
Bertalanffy growth curves (Figure @ref(fig:bioinputs)A).

\hypertarget{model-parameters}{%
\subsection{MODEL PARAMETERS}\label{model-parameters}}

\hypertarget{natural-mortality}{%
\subsubsection{Natural mortality}\label{natural-mortality}}

Natural mortality \emph{M} was assumed constant over time and age and
fixed at 0.10 (Johnson and Quinn 1988). Code infrastructure has been
developed to estimate \emph{M} using a prior as is done in the federal
assessment, but this methodology will not be implemented until prior
distributions can be thoroughly analyzed.

\hypertarget{discard-mortality}{%
\subsubsection{Discard mortality}\label{discard-mortality}}

Stachura et al.~(2012) estimated discard mortality \emph{D} of sablefish
to be 11.7\% using release--recapture data from a longline survey in
Southeast Alaska. It is likely that discard mortality in a fishery is
higher due to careful fish handling on survey vessels during tagging
experiments. Therefore, the discard mortality rate from the Pacific
halibut fishery, \emph{D}=16\%, was used (Gilroy and Stewart 2013). The
Pacific halibut fishery is assumed a reasonable proxy for sablefish
because the fisheries utilize similar gear and frequently the same
vessels and crew participate in both fisheries. Moreover, both species
are considered hardy and do not experience barotrauma.

\hypertarget{selectivity}{%
\subsubsection{Selectivity}\label{selectivity}}

The longline fishery and survey are assumed to follow a logistic
selectivity pattern. The current parameterization of the logistic curves
uses \emph{\(s_{50}\)} and \emph{\(\delta\)}, which represent the ages
at which 50\% of fish are selected by the gear (\emph{\(s_{50}\)}) and
the shape or slope of the logistic curve (\emph{\(\delta\)}).
Selectivity-at-age (\emph{\(s_a\)}) for this parameterization is defined
as

\[
\begin{equation}
s_a=\frac{1}{1+\mbox{exp}(-k(a-s_{50}))}.
(\#eq:sel2)
\end{equation}
\]

Selectivity is fit separately for the longline fishery (\emph{fsh}) and
survey (\emph{srv}). There is flexibility to define discrete time blocks
for both fishery and survey selectivity.

Currently, fishery selectivity is fixed in the model using federal
selectivity values for the derby (pre-EQS) and contemporary fishery
(EQS) (Goethel et al.~2022; Figure @ref(fig:slx)). Estimating
selectivity is challenging when accounting for fishery releases because
no age or length data are available on the released fish. Further
research is needed to better characterize how discarding behavior has
changed over time and if discarding was common pre-EQS.

Selectivity in the longline survey is now estimated in the model using
two time blocks representing the unstandardized survey (pre-2000) and
the fully standardized survey that began in 2000.

\hypertarget{catchability}{%
\subsubsection{Catchability}\label{catchability}}

Currently 5 parameters for catchability are estimated: 2 for fishery
catchability (pre-EQS and EQS) ln(\emph{\(q_{fsh}\)}), 2 for the ADF\&G
longline survey ln(\emph{\(q_{srv}\)}), and 1 for the mark--recapture
abundance index ln(\emph{\(q_{MR}\)}).

\hypertarget{recruitment-and-initial-numbers-at-age}{%
\subsubsection{Recruitment and initial
numbers-at-age}\label{recruitment-and-initial-numbers-at-age}}

The numbers-at-age matrix \emph{N} is parameterized with mean
log-recruitment \emph{\(\mu_R\)}, 48 (\emph{T}) log recruitment
deviations \emph{\(\tau\)}, mean log initial numbers-at-age
\emph{\(\mu_N\)}, and 28 (\emph{A} − 2) deviations from mean log initial
numbers-at-age \emph{\(\psi\)}. The parameter that governs the
variability in \emph{\(\tau\)} and \emph{\(\psi\)},
ln(\emph{\(\sigma_R\)}), is estimated within the model using random
effects.

\hypertarget{fishing-mortality}{%
\subsubsection{Fishing mortality}\label{fishing-mortality}}

There is 1 parameter estimated for mean log-fishing mortality,
\emph{\(\mu_F\)}, and 48 (\emph{T}) log-fishing mortality deviations
\emph{\(\phi\)}.

\hypertarget{population-dynamics}{%
\subsection{POPULATION DYNAMICS}\label{population-dynamics}}

The population dynamics of this model are governed by the following
state dynamics equations, where the number of sablefish \(N\) in year
\(t=1\), age \(a\), and sex \(k\) are defined as

\[
\begin{equation}
N_{1,a,k} = \left\{ \begin{array}{ll}
0.5\cdot\mbox{exp}(\mu_R-M(a-a_0)+\psi_a) &a_{0}<a<a_{+}\\
0.5\cdot\mbox{exp}(\mu_R-M(a_{+}-1))/(1-\mbox{exp}(-M)) &a=a_{+}
\end{array}\right.
(\#eq:Nmat1)
\end{equation}
\]

Recruitment to age-2 in all years and the remaining projected \(N\)
matrix is defined as

\[
\begin{equation}
N_{t,a,k} = \left\{ \begin{array}{ll}
0.5\cdot\mbox{exp}(\mu_R+\tau_t) &a=a_0\\
0.5\cdot N_{t-1,a-1,k}\mbox{exp}(Z_{t-1,a-1,k}) &a_{0}<a<a_{+}\\
0.5\cdot N_{t-1,a-1,k}\mbox{exp}(Z_{t-1,a-1,k})+N_{t-1,a,k}\mbox{exp}(Z_{t-1,a,k}) &a=a_{+}
\end{array}\right.
(\#eq:Nmat2)
\end{equation}
\]

where the total instantaneous mortality, \(Z_{t,a,k}\), is the sum of
natural mortality \(M\) and fishing mortality \(F_{t,a,k}\). Sex ratios
are assumed 50/50 at time of recruitment, thus any changes in sex ratios
in the population over time are the result of sex-specific, fully
selected fishing mortality. Total annual fishing mortality
\emph{\(F_t\)} is defined as

\[
\begin{equation}
F_t=\mbox{exp}(\mu_F+\phi_t).
(\#eq:fmort1)
\end{equation}
\]

Fishing mortality is modeled as a function of fishery selectivity
\(s_{t,a,k}\), retention probability \(R_{a,k}\) (the age-specific
probability of being landed given being caught, Figure
@ref(fig:retention)), and discard mortality \(D\):

\[
\begin{equation}
F_{t,a,k}=s_{t,a,k}^{fsh}(R_{a,k}+D(1-R_{a,k}))F_t.
(\#eq:fmort2)
\end{equation}
\]

\hypertarget{predicted-values}{%
\subsection{PREDICTED VALUES}\label{predicted-values}}

Predicted fishery CPUE (kg per hook) in year \(t\), \(\hat{I_t}^{fsh}\),
is defined as a function of fishery catchability \(q_{fsh}\) and biomass
available to the fishery:

\[
\begin{equation}
\hat{I_t}^{fsh}=q_{fsh}\sum_{k=1}^{2}\sum_{a=a_0}^{a+}w_{a,k}^{srv} \cdot s_{t,a,k}^{fsh} \cdot N_{t,a,k} \cdot S^{fsh},
(\#eq:predfshcpue)
\end{equation}
\]

where \(w_{a,k}^{srv}\) is mean weight-at-age by sex in the longline
survey. Survival (\(S^{srv}\)) to the beginning of the fishery in August
is defined as

\[
\begin{equation}
S_{t,a,k}^{fsh}=exp(-\frac{8}{12}(M+F_{t,a,k}))
(\#eq:bigS)
\end{equation}
\]

Survival equations include natural and fishing mortality because the
model assumes continuous fishing mortality.

Predicted longline survey CPUE (numbers per hook) in year \(t\),
(\(\hat{I_t}^{srv}\)) is defined as a function survey catchability
\(q^{srv}\), abundance available to the survey, and survival to the
beginning of the survey in July (\(S^{srv}\)):

\[
\begin{equation}
\hat{I_t}^{srv}=q_{srv}\sum_{k=1}^{2}\sum_{a=a_0}^{a+}s_{t,a,k}^{srv} \cdot N_{t,a,k} \cdot S^{srv}.
(\#eq:predsrvcpue)
\end{equation}
\]

Predicted mark-recapture abundance in year \(t\) (\(\hat{I_t}^{MR}\)) is
defined as a function of mark-recapture catchability \(q^{MR}\),
abundance available to the fishery, and survival to the beginning of the
NSEI fishery in August (\(S_{t,a,k}^{fsh}\)):

\[
\begin{equation}
\hat{I_t}^{MR}=q_{MR}\sum_{k=1}^{2}\sum_{a=a_0}^{a+}s_{t,a,k}^{fsh} \cdot N_{t,a,k} \cdot S_{t,a,k}^{fsh}.
(\#eq:predmr)
\end{equation}
\]

Spawning biomass \(SSB\) is calculated as

\[
\begin{equation}
SSB=\sum_{a=a_0}^{a+} w_{a,f}^{srv} \cdot N_{t,a,f} \cdot S_{t,a,k}^{spawn} \cdot p_a,
(\#eq:ssb)
\end{equation}
\]

where \(w_{a,f}^{srv}\) is mean weight-at-age of females in the longline
survey, \(S^{spawn}\) is the fraction of individuals surviving to spawn
in February, and \(p_a\) is the proportion of mature females-at-age. In
the single sex model, proportion of females at age in the survey \(r_a\)
is used to get the female portion of the \(N\) matrix.

Predicted survey age compositions (sexes combined) are computed as

\[
\begin{equation}
\hat{P}_{t,a}^{srv}=\Omega_{a',a}\frac{\sum_{k=1}^{2}N_{t,a,k} \cdot s_{a,k}^{srv}}{\sum_{k=1}^{2}\sum_{a=a_0}^{a+} N_{t,a,k} \cdot s_{a,k}^{srv}},
(\#eq:predsrvage)
\end{equation}
\]

where \(\Omega_{a',a}\) is the ageing error matrix. Predicted fishery
age compositions (sexes combined) are computed as

\[
\begin{equation}
\hat{P}_{t,a}^{fsh}=\Omega_{a',a}\frac{\sum_{k=1}^{2}C_{t,a,k}}{\sum_{k=1}^{2}\sum_{a=a_0}^{a+} C_{t,a,k}},
(\#eq:predfshage)
\end{equation}
\]

where \(\hat{C}_{t,a,k}\) is the predicted landed catch in
numbers-at-age by sex derived from a modified Baranov catch equation:

\[
\begin{equation}
\hat{C}_{t,a,k}=N_{t,a,k}\frac{R_{a,k}F_{t,a,k}}{Z_{t,a,k}}(1-\mbox{exp}(-Z_{t,a,k})),
(\#eq:landed)
\end{equation}
\]

where \(R_{a,k}\) is the assumed probability of retention by age and sex
(Figure @ref(fig:retention)).

Predicted landed catch in biomass \(\hat{Y}\) is calculated as the
product of fishery weight-at-age, \(w_{a,k}^{fsh}\), and landed catch in
numbers-at-age:

\[
\begin{equation}
\hat{Y}_t=\sum_{k=1}^{2}\sum_{a=a_0}^{a+} w_{a,k}^{fsh} \cdot \hat{C}_{t,a,k}.
(\#eq:yield)
\end{equation}
\]

The predicted biomass of discarded sablefish estimated to die (\(W_t\))
with an assumed discard mortality (\(D\)) of 0.16 is

\[
\begin{equation}
\hat{W}_t= \sum_{k=1}^{2}\sum_{a=a_0}^{a+}w_{a,k}^{srv}N_{t,a,k}\frac{D (1-R_{a,k})F_{t,a,k}}{Z_{t,a,k}}(1-\mbox{exp}(-Z_{t,a,k})).
(\#eq:wastage)
\end{equation}
\]

Predicted survey length compositions are calculated using the
sex-specific age-length keys (\(\Lambda_{a,l,k}\)), such that

\[
\begin{equation}
\hat{P}_{t,l,k}^{srv}=\Lambda_{a,l,k}\frac{N_{t,a,k} \cdot s_{a,k}^{srv}}{\sum_{a=a_0}^{a+} N_{t,a,k} \cdot s_{a,k}^{srv}}.
(\#eq:predsrvlen)
\end{equation}
\]

Predicted fishery length compositions are calculated as

\[
\begin{equation}
\hat{P}_{t,l,k}^{fsh}=\Lambda{a,l,k}\frac{\hat{C}_{t,a,k}}{\sum_{a=a_0}^{a+} \hat{C}_{t,a,k}}.
(\#eq:predfshlen)
\end{equation}
\]

\hypertarget{biological-reference-points}{%
\subsection{BIOLOGICAL REFERENCE
POINTS}\label{biological-reference-points}}

Biological reference points for NSEI sablefish were developed for the
SCAA model following the federal assessment ADMB code (D. Hanselman,
Fisheries Research Biologist, NOAA, Juneau, personal communication,
April 2019). They are based on spawning potential ratio (SPR), or the
average fecundity of a recruit over its lifetime divided by the average
fecundity of a recruit over its lifetime when the stock is unfished.
Spawning stock biomass is used as a proxy for fecundity, which assumes
that weight-at-age and fecundity-at-age are proportionally related.

The theoretical numbers-at-age per recruit (\(N_a^{SPR}\)) under the
current harvest policy \(F_{50}\) (the fishing mortality that results in
a SPR of 50\%) is initialized with 1, then populated assuming the most
recent year's values (\emph{T}) for female fishery selectivity-at-age
and estimated \(F_{50}\):

\[
\begin{equation}
N_{a}^{SPR50} = \left\{ \begin{array}{lll}
1 &a = a_{0}\\
N_{a-1}^{SPR50}exp(-M-F_{50}s^{fsh}_{a-1,fem}) &a_{0} < a < a_{+})\\
N_{a-1}^{SPR50}exp(-M-F_{50}s^{fsh}_{a-1,fem}) +  N_{a}^{SPR50}exp(-M-F_{50}s^{fsh}_{T,a,fem}) &a=a_{+}
\end{array}\right.
(\#eq:Nspr50)
\end{equation}
\] The \(N_a^{SPR}\) under unfished conditions (relating to an SPR of
100\%) collapses to

\[
\begin{equation}
N_{a}^{SPR100} = \left\{ \begin{array}{lll}
1 &a = a_{0}\\
N_{a-1}^{SPR100}exp(-M) &a_{0} < a < a_{+})\\
N_{a-1}^{SPR100}exp(-M) +  N_{a}^{SPR100}exp(-M) &a=a_{+}
\end{array}\right.
(\#eq:Nspr100)
\end{equation}
\] The spawning biomass per recruit (\({SPBR}_{SPR}\)) under fished
(e.g., \emph{SPR} = 50\%) and unfished (\emph{SPR} = 100\%) condition is

\[
\begin{equation}
SPBR_{SPR}={\sum_{a=a_0}^{a+}}w_{a,f}^{srv}*N_a^{spr}*S_{T,a,f}^{spawn}*p_a.
(\#eq:SPBRSPR)
\end{equation}
\] Equilibrium recruitment is assumed to be equal to the geometric mean
of the full estimated recruitment time series such that

\[
\begin{equation}
\dot{R}=({\prod_{t=1}^{T}exp(\mu_R+\tau_t)})^{\frac{1}{T}}.
(\#eq:SPBRSPR)
\end{equation}
\] This assumption differs from the federal model, which assumes the
arithmetic mean instead of the geometric mean. The geometric mean is a
more appropriate measure of central tendency because sablefish
recruitment is best described by a multiplicative function. Using the
arithmetic mean in this case results in an equilibrium value for
recruitment that is biased high.

Assuming a 50/50 sex ratio for recruitment, equilibrium female spawning
biomass (\({SB}_{SPR}\)) under fished and unfished conditions is
calculated as

\[
\begin{equation}
SB_{SPR}=0.5*\dot{R}*SPBR_{SPR}
(\#eq:SBSPR)
\end{equation}
\] The SPR-based fishing mortality rate of \(F_{50}\) is estimated using
penalized likelihood, where

\[
\begin{equation}
lnL(SPR) = 100({\frac{SBPR_{50}}{SBPR_{100}}-0.50})^2
(\#eq:lnSPR)
\end{equation}
\] In addition to \(F_{50}\), \(F_{35}\), \(F_{40}\), \(F_{60}\) and
\(F_{70}\) are estimated fro comparison.

The maximum permissible ABC is calculated as the difference between the
predicted landed proportion of the catch (\(\hat{Y}_{T+1}\)) and the
estimated mortality from releases (\(\hat{W}_{T+1}\)) under \(F_{50}\)
using forecasted estimates of abundance (\(N_{T+1}\)). Equation details
for \(\hat{Y}_{T+1}\) and \(\hat{W}_{T+1}\) are detailed in the section
of this report titled ``Predicted Values.''

\hypertarget{likelihood-components}{%
\subsection{LIKELIHOOD COMPONENTS}\label{likelihood-components}}

The objective function, or the total negative log-likelihood to be
minimized, included the sum of the following likelihood components \(L\)
which received individual weights \(\lambda\):.\\

Landed catch biomass (\(Y\)) was modeled using a lognormal likelihood
where \(\sigma_Y\) was assumed to be 0.05:

\[
\begin{equation}
\mbox{ln}L(Y)=\lambda_Y\frac{1}{2\sigma_Y^2}\sum_{t=1}^{T}\Big(\mbox{ln}(Y_t+c)-\mbox{ln}(\hat{Y}_t+c)\Big)^2 ,
(\#eq:catchlike)
\end{equation} 
\] where \(\lambda_Y\) = 1.0 and \(c\) is a small constant set at 0.0001
to allow approximately zero catches in log-space.

Fishery CPUE, survey CPUE, and the mark-recapture abundance index were
modeled using lognormal likelihoods, where \(\sigma_I\) was assumed to
be 0.08 for the fishery and survey CPUEs and 0.5 for the mark-recapture
abundance index:

\[
\begin{equation}
\mbox{ln}L(I)=\lambda_I\frac{1}{2\sigma_I^2}\sum_{t=1}^{T_I}\Big(\mbox{ln}(I_t+c)-\mbox{ln}(\hat{I}_t+c)\Big)^2 ,
(\#eq:indexlike)
\end{equation} 
\] where \(T_I\) is the number of years of data for each index and
\(\lambda_I\) is set to 1.0.

Fishery and survey age compositions were modeled using the multinomial
likelihood (\(P^{age}\)), where inital effective sample size
\(\omega_t\) was calculated as the square root of the total sample size
in year \(t\):

\[
\begin{equation}
\mbox{ln}L(P^{age})=\lambda_{P^{age}}\sum_{t=1}^{T_P^{age}} - \omega_t \sum_{a=a_0}^{a+} (P_{t,a}+c)\cdot\mbox{ln}(\hat{P}_{t,a}+c),
(\#eq:agemult)
\end{equation} 
\] where \(T_P^{age}\) is the number of years of data for each age
composition, \(\lambda_{P^{age}}\) is set to 1.0, and \(c\) prevents the
composition from being 0 in the likelihood calculation. The effective
sample size used in the final model was estimated using iteritive
re-weighting methods described by McAllister and Ianelli (1997).

The Dirichlet-multinomial likelihood is also available in the SCAA code,
which derives effective sample size through the estimation of an
additional parameter \(\theta\) using the Dirichlet-multinomial
likelihood (Thorson et al.~2017):

\[
\begin{equation}
\mbox{ln}L(P^{age})=\sum_{t=1}^{T_P^{age}} -\Gamma(n_t+1)-\sum_{a=a_0}^{a+}\Gamma(n_t P_{t,a}+1)+\Gamma(n_t\theta)-\Gamma(n_t+\theta n_t)+\sum_{a=a_0}^{a+}\Big[\Gamma(n_tP_{t,a}+\theta n_t \hat{P}_{t,a})-\Gamma(\theta n_t \hat{P}_{t,a})\Big],
(\#eq:agedirich)
\end{equation} 
\] where \(n\) is the input sample size. The relationship between \(n\),
\(\theta\), and \(\omega\) is

\[
\begin{equation}
\omega_t = \frac{1+\theta n_t}{1+\theta}.
(\#eq:effn)
\end{equation} 
\] Further exploration is needed to implement the Dirichlet-multinomial
as efforts on this assessment failed to reach convergence when the
Dirichlet-multinomial was implemented. As such only results for the
multinomial likelihood tuned using McAllister and Ianelli (1997) are
presented in the current assessment.

Fishery and survey length compositions by sex are modeled using the
multinomial likelihood (\(P^{len}\)), where initial effective sample
size \(\omega_t\) is calculated as the square root of the total sample
size in year \(t\): \[
\begin{equation}
\mbox{ln}L(P^{len})=\lambda_{P^{len}}\sum_{k=1}^{2}\sum_{t=1}^{T_P^{len}} - \omega_t \sum_{l=l_0}^{l+} (P_{t,l}+c)\cdot\mbox{ln}(\hat{P}_{t,l}+c).
(\#eq:lenmult)
\end{equation} 
\] \(T_P^{len}\) is the number of years of data for each length
composition and \(\lambda_{Plen}\) is set to 1.0. The
Dirichlet-multinomial likelihood is also available for length
compositions but failed to converge for this assessment. As such the
multinomial likelihoods tuned using McAllister and Ianelli (1997) are
used in this assessment.

Annual log-fishing mortality deviations (\(\phi_t\)) are included with a
penalized lognormal likelihood, where \[
\begin{equation}
\mbox{ln}L(\phi)=\lambda_{\phi}\sum_{t=1}^{T}\phi_t^2,
(\#eq:fmortlike)
\end{equation} 
\] where \(\lambda_{\phi}\)=0.1.

Recruitment deviations (\(\tau_t\)) are modeled using random effects
such that \[
\begin{equation}
\mbox{ln}L(\tau)=\lambda_{\tau}\sum_{i=1}^{T+A-2}\mbox{ln}(\sigma_R)+\frac{(\tau_i-0.5\sigma_R^2)^2}{2\sigma_R}.
(\#eq:randomrec)
\end{equation} 
\] where -0.5\(\sigma^2\) is a bias correction needed to obtain the
expected value (mean) instead of the median, and \(\lambda^{\tau}\) is
fixed to 2.0. The initial numbers-at-age deviations \(\psi_a\) are
implemented in the same way as recruitment deviations and are governed
by the same \(\sigma_R\) . Unlike ADMB, TMB allows fast implementation
of nonlinear random effects models by estimating the marginal likelihood
of the fixed effects via the Laplace approximation and estimating the
random effects using empirical Bayes methods (Kristensen et al.~2016).

\hypertarget{priors}{%
\subsubsection{Priors}\label{priors}}

Because the mark--recapture abundance index scales the exploitable
population, a normal prior is imposed on \(q_{MR}\) of 1.0 with a
standard deviation of 0.1. Vague priors are assigned to fishery and
survey \emph{q}. Future work on this model should include the
development of priors for fishery and survey \emph{q}.

\hypertarget{model-results}{%
\subsection{MODEL RESULTS}\label{model-results}}

A total of 146 parameters were estimated in the SCAA model, which
converged with a maximum gradient component less than 0.001 (Table
@ref(tab:keyparams)). The objective function value (negative log
likelihood) was 1799 (Table @ref(tab:likesum)). The model fits catch,
survey CPUE, and pre-EQS fishery CPUE reasonably well in most years
(Figure @ref(fig:predabdind)). Contemporary fishery CPUE (EQS) does not
fit well, with long runs of positive or negative residuals (Figure
@ref(fig:predabdind)B). The model performs poorly during the period
directly following the implementation of EQS in 1994 for all indices,
including catch (Figure @ref(fig:predabdind)). Additionally, the fit to
the mark-recapture abundance estimates have worsened with the model
estimating higher abundance than indicated the mark-recapture project in
earlier years, although it fits well in recent years (Figure
@ref(fig:predabdind)D).

Further consideration should be given to which abundance indices should
be used in the model. For example, because releasing fish is legal in
NSEI and past logbook data have not required released fish to be
recorded, fishery CPUE may not be a suitable index of abundance.
Starting in 2019, fishermen were required to provide an estimated number
of released sablefish by set; however, there is no record of length or
weight of these releases.

The mark-recapture estimate of abundance is also likely biased to some
degree and overestimates precision. The project relies on tag returns
from the fishery and tag accounting rarely matches the count of fin
clips at processor plants performed by ADF\&G staff. Under and/or over
reporting of tag recoveries likely biases the results to some degree and
the bias may be different from year to year depending on retention
incentives. Furthermore, the removal of tags by fisherman prior to exam
by ADF\&G staff prevents the ability to identify and correct for tag
loss. Lastly, the current mark-recapture analysis does not correct for
size or geographic differences in capture probabilities, which will bias
results to some degree. Examining these sources of biases remains a
priority.

Finally, variability in catch, survey and fishery CPUE indices, and the
mark--recapture abundance estimate was assumed. Future enhancements
could include estimating this variability using available data and
allowing the SCAA model to estimate extra variance based on the fit to
the entire data set.

Derived indices of age-2 recruitment, female spawning stock biomass, and
exploitable abundance and biomass (i.e., available to the fishery)
suggest that this stock has been in a period of low productivity since
the mid-1990s but has experienced a surge of recruitment in recent
years, highlighted by the strong 2016-year class (Figure
@ref(fig:derivedts)). Recruitment trends are comparable with federal
values, and estimates of spawning stock biomass, exploitable biomass,
and exploitable abundance, including large recruitment events (Goethel
et al.~2022; Sullivan et al.~2019). Although recruitment has been strong
in recent years and biomass is clearly expanding as these fish grow and
mature, the population remains below historical levels evident in the
early 90's. And while the dominance of the younger age classes is the
result of these strong recruitment events, the lack of older sablefish,
which can live into their 90s, remains concerning given the likely
outside contribution these older fish make to the spawning population.

Fits to the age composition data is improved from past assessments,
however still fails to capture all of the variability in the data
(Figure @ref(fig:fshage) and @ref(fig:srvage)). Although the model fits
the general shape of the age compositions in most years, there are poor
residual patterns (Figure @ref(fig:residage)). Additionally, the model
appears to underestimate fits to the plus group ages, which should be
explored in future assessments.

Fits to the length composition data also remain poor and suffer from
poor residual patterns signifying that the model is underestimating
smaller, mid-size classes and overestimating larger and the smallest
size classes (Figures @ref(fig:malefshlen), @ref(fig:femalefshlen),
@ref(fig:malesrvlen), @ref(fig:femalesrvlen) and @ref(fig:residlen)).
Like the age compositions, the model predicts the general shape of the
length compositions for both the survey and fishery in most years.
Despite this, there are also poor residual patterns in the length
compositions, and the model is not predicting the small individuals
observed in the survey in recent years.

The lack of fit to the age and length composition data likely results
from restrictions of fishery and survey selectivity in the model. Survey
selectivity is now estimated in the model, which appears to have
improved model fit. However, survey selectivity is modeled in two time
blocks and allowing time-varying survey selectivity may further improve
fit to the data. Fishery selectivity is further restricted as the values
are fixed to the federal model values owing to an inability in the model
to estimate it. Because no data on fishery releases exist, it may not be
possible to estimate fishery selectivity that fit to the composition
data. Stock assessments that account for discarded catch frequently have
observer data and will overcome this challenge through the estimation of
a separate selectivity curve for discarded catch (e.g., Zheng and
Siddeek 2018). Methods to improve fits to fishery composition data
should be developed in future assessments, including modeling changes in
retention probability over time using price per pound and catch
composition data. It may also be possible to loosen reliance on the
federal curves by placing prior around the selectivity parameters rather
than fixing those values.

Changes made to the operating model resulted in lower estimates of stock
status although the overall trajectory of the stock remains the same.
Tuning the model to estimate the effective sample sizes or the age and
length composition data placed more weight on the composition data and
had the effect of increasing the biomass estimates (Table
@ref(tab:modcomp)). Updating the selectivity curves to the most recent
values in the federal assessment resulted in lower biomass estimate,
although still above the base model using the old selectivity estimates.
The updates made for model v23 that include estimating survey
selectivity within the model and estimating recruitment deviations using
random effects resulted in lower biomass estimates. The population still
appears to be increasing, however the fishery appears closer to the
management target of \(SPR_{50}\) than estimated by the base model.

Estimation of recruitment deviations using random effects produced much
lower values of \(\sigma_R\) than had been fixed to the federal model
value of 1.2 (Table @ref(tab:keyparams)). The federal value is
noticeably higher than that estimated for other Alaska groundfish stocks
(Lynch et al.~2018; Hanselman et al.~2019) whereas the estimate from
model v23 was much more in line with other Alaska groundfish at 0.52.

Despite challenges to fitting the data, the model demonstrates good
retrospective patterns. Retrospective patterns are defined as
``systematic changes to estimates of population size, or other
assessment model-derived quantities, that occur as additional years of
data are added to, or removed from, a stock assessment'' (Hutado-Ferro
et al.~2015). They cause over- or underestimation of stock size, which
can lead to flawed harvest recommendations or management advice. A
positive retrospective pattern or bias can result in overestimation of
stock biomass, which if persistent over many years, will result in the
realized fishing mortality rate exceeding the target harvest policy
(i.e., overfishing). Alternatively, a persistent negative retrospective
pattern or bias will translate into foregone yields and fishing
opportunity.

\hypertarget{retrospective-analysis}{%
\subsubsection{Retrospective analysis}\label{retrospective-analysis}}

Following recommendations from the North Pacific Fishery Management
Council's Groundfish Plan Team (Hanselman et al.~2013), a retrospective
analysis was performed by dropping the last 10 years of data (i.e.,
``peels''), plotting spawning biomass, fishing mortality, and
recruitment time series for each model run, and plotting the relative
changes in reference to the terminal model (2022). Mohn's \(\rho\) was
calculated for spawning biomass, fishing mortality and recruitment such
that

\[
\begin{equation}
\text{Mohn's } \rho=\sum_{p=1}^{P}\frac{X_{Y-p,p}-X_{Y-p,0}}{Y_{Y-p,0}}/P
(\#eq:mohns)
\end{equation} 
\] where \emph{Y} is the last year in the full time series, \emph{p} is
the number of years at the end of the peeled data series, and \emph{X}
denotes the estimate of the quantity of interest (i.e., spawning
biomass, fishing mortality or recruitment)(Mohn 1999; Hanselman et
al.~2013).

Model v23 demonstrates a small, positive bias in spawning biomass
(Mohn's \(\rho\) = 0.05; Figure @ref(fig:mohnsbiom)) and a slight
negative bias in fishing mortality (Mohn's \(\rho\) = -0.03) that are
well within the acceptable range for a long-lived groundfish species.
There is a larger positive bias in Age-2 recruits (Mohn's \(\rho\) =
0.10, Figure @ref(fig:mohnsrec)), however, individual years may over or
underestimate recruitment by up to 200\%. It should be noted that the
model tends to overestimate recruitment when recruitment is low and
underestimate recruitment when recruitment is high. That is to say, in
recent years that have shown clear signs of high recruitment, the model
tends to underestimate those year classes.

\hypertarget{marking-survey-sensitivity-analysis}{%
\subsubsection{Marking survey sensitivity
analysis}\label{marking-survey-sensitivity-analysis}}

The mark--recapture project has formed the foundation of sablefish
management in NSEI since 2005 and the abundance estimate provides a
snapshot of the exploitable abundance in NSEI (Figure @ref(fig:abdind)D;
Dressel 2009). There are numerous shortcomings to the mark-recapture
project which are detailed elsewhere in this report and the abundance
estimates certainly overestimate precision and are likely biased to some
unknown degree that likely varies in direction and strength over the
course of the time series. Due to budget constraints the mark-recapture
project does not occur every year and uncertainty with future funding
was part of the impetus for adopting the SCAA which is less reliant on
yearly abundance estimates (Sullivan et al.~2020). With the adoption of
the SCAA model, an initial analysis was performed to determine the
effects of performing the mark-recapture project every other, or every
third year, and the model was found to perform adequately under those
circumstances (Sullivan et al.~2020).

There continues to be interest in possibly abandoning the mark-recapture
project all together owing to it's expense and the amount of staff time
required to enact the project. In this year's assessment we examined
simpler scenarios than examined by Sullivan et al.~(2020) and simply
dropped the last 5 and 10 year's of mark-recapture data from the model
to determine how ABC's and spawning biomass would compare to the full
data set with model v23 (Table @ref(tab:modcomp)). If there had been no
mark-recapture project in the last 5 years, the maximum ABC and the
estimated age-2 biomass would be 0.9\% higher. Had there been no
mark-recapture project in the last 10 years, the maximum ABC would have
been 12.1\% higher and the estimated age-2 biomass would be 10.4\%
higher.

These results, combined with Sullivan et al's (2020) analysis, continue
to demonstrate that this assessment will produce consistent results when
the mark-recapture project is not performed every year. However, it is
important to note that the other indices of abundance, survey CPUE and
fishery CPUE, fail to provide any scale to the population and the
mark-recapture abundance estimate is the only data source that anchors
the model to an estimate of true abundance. If the mark-recapture
project was completely abandoned, the assessment would not likely
deprecate in the first several years, however, over time the estimates
of biomass and associated biological reference points are likely to
drift away from what the true biomass might be. While it remains
important to revisit the mark-recapture analysis to estimate and
potentially correct biases in the abundance estimates, it is also
important to recognize this data as a key piece of information for the
assessment if time, staffing and funding remain available.

\hypertarget{abc-recommendations}{%
\subsection{ABC RECOMMENDATIONS}\label{abc-recommendations}}

The recommended ABC for 2023 is derived from an average of the
recommended ABC from the base model and model v23. Regardless of model
choice, the population continues to expand with the growth and
maturation of the 2013-2018 year classes. Model v23 shows the population
to be much closer to SB50 than does the base model (Figure
@ref(fig:Fsprcomp)) and using this model would result in an increase in
the ABC of 3\% from last year. Given that the population is increasing,
and the population is forecast to continue increasing in the next
several years (albeit, at a slowing rate) we felt that in fairness to
the fleet that averaging the two models was appropriate. This may result
in a small to negligible change in the ABC in the 2024 assessment as
model development continues and model v23 becomes the base model for the
next assessment.

Model v23 results in a maximum permissible ABC (max ABC) of 1,486,406
round lbs at the target fully selected fishing mortality of \(F_{50}\)
(Table @ref(tab:brps)). This is a 43,092 round lb increase (3\%) from
the 2022 ABC of 1,443,314 round lbs. The base model produces a max ABC
of 1,873,598 round lbs (30\% higher than last year's max ABC) which
under the max 15\% change would have resulted in a recommended ABC of
1,659,811 round lbs (or a 15\% increase). Balancing model v23 with the
base model and averaging the recommended ABC from the two models results
in a recommended ABC of 1,573,109 round lbs, of a 9\% increase from last
year's ABC. Mortality from fishery releases under \(F_{50}\), assuming
fixed retention probabilities and a discard mortality of 0.16 is
estimated to be 69,522 lbs in model v23 and 79,711 in the base model,
which was included in the max ABC calculation (Table @ref(tab:brps) and
@ref(tab:decrements)).

While there is uncertainty in the absolute estimate of sablefish biomass
in the NSEI, the population is undoubtedly increasing as the 2013-2018
year classes continue to grow and mature. This trend is likely to
continue over the next several years as these fish become fully mature
and reach maximum size. While this is certainly a good sign, it is
important to note that the population remains below historical levels
and that there is still a lack of older fish in the population. Older
females likely contribute disproportionally to the spawning output in
the population and it remains desirable to maintain fishing pressure
that allows the younger age classes to grow and mature.

\hypertarget{future-work-and-recomendations}{%
\section{FUTURE WORK AND
RECOMENDATIONS}\label{future-work-and-recomendations}}

These tasks are viewed as the next steps in developing the SCAA:

It is expected that participation in the pot fishery in 2023 will
increase dramatically as it has in the SSEI and the federal fishery
where pots have been legal for several years. This will need to be
monitored closely to see how catch rates and fish size varies between
the longline and pot fisheries. This issue will involve significant
model development and will be of primary concern as the fleet changes
fishing practices.

Develop methods to estimate fishery selectivity as this will make the
model less dependent on federal values and the assumption that
selectivity in the federal fishery mirrors that in the NSEI fishery.
Initial efforts to do this failed to produce converged numbers and
reasonable estimates of selectivity. Exploring the use of priors on the
selectivity parameters, based on the federal estimates, may be an
option. Exploring time varying selectivity in both the fishery and the
survey may also provide options that could improve the fit of age and
length data.\\

Review the mark-recapture analysis for two primary reasons:

Determine if less biased estimates of abundance can be produced and by
modelling size and geographic differences in capture probabilities, and

Determine the level of bias in the abundance estimates by comparing
recapture rates between the longline survey and the fishery

Continue to develop proper data weighting for the model by

using estimated uncertainty in the indices and allowing the model to
estimate extra-uncertainty parameters, and

continuing to develop the Dirichlet data weighting of the age and length
composition data.

Implement the SCAA model in a Bayesian framework. Preliminary work has
been done using the R library tmbstan (Monnahan and Kristensen 2018).
The process is currently very slow; the next steps include optimizing
the NUTS algorithm using methods detailed in the supplementary material
of Monnahan and Kristensen (2018).

\hypertarget{acknowledgements}{%
\section{ACKNOWLEDGEMENTS}\label{acknowledgements}}

Many thanks to ADF\&G Region I Groundfish Project staff who have
collected NSEI sablefish data, maintained documentation, and worked to
improve the conservation and management of this unique fishery.
Additionally, we would like to thank the Age Determination Unit staff,
including Kevin McNeel, Chris Hinds, and Catherine Mattson, who provide
age data in a timely manner for stock assessments. We are grateful to
Region I analyst/programmers, Karl Wood and Justin Daily, who provide
database support and application development. Thans to Jane Sullivan,
the original author of this model, who continues to offer insight and
advice in developing this assessment. We also wish to thank Curry
Cunningham and his graduate student Matt Cheng with the University of
Alaska Fairbanks, College of Fisheries and Ocean Science for their
advice in developing this model. Finally, we are thankful to the NOAA
scientists for their continued collaborations and sharing their
sablefish knowledge.

\hypertarget{references}{%
\section{REFERENCES}\label{references}}

Akaike, H. 1974. A new look at the statistical model identification.
IEEE Transactions on Automatic Control 19:716--723.

Beder, A., J. Stahl. 2016. Northern Southeast Inside Commercial
Sablefish Fishery and Survey Activities in Southeast Alaska, 2015.
Alaska Department of Fish and Game, Fishery Management Report No.~15-27,
Anchorage, Alaska.

Carlile, D. W., Richardson, B., Cartwright, M., and O'Connell, V.M.
2002. Southeast Alaska sablefish stock assessment activities 1988--2001,
Alaska Department of Fish and Game, Division of Commercial Fisheries
Juneau, Alaska.

Chapman, D. G. 1951. Some properties of the hypergeometric distribution
with applications to zoological census. University of California
Publications in Statistics 1:131--160.

Depaoli, S., James P. Clifton, and Patrice R. Cobb. 2016. Just Another
Gibbs Sampler (JAGS) Flexible Software for MCMC Implementation. Journal
of Educational and Behavioral Statistics 41.6: 628-649.

Dressel, S.C. 2009. 2006 Northern Southeast Inside sablefish stock
assessment and 2007 forecast and quota. Alaska Department of Fish and
Game, Fishery Data Series No.~09-50, Anchorage, Alaska.

Echave, K. B., D. H. Hanselman, M. D. Adkison, M. F. Sigler. 2012.
Inter-decadal changes in sablefish, Anoplopoma fimbria, growth in the
northeast Pacific Ocean. Fish. Bull. 210:361-374.

Fournier, D. and C. P. Archibald. 1982. A general theory for analyzing
catch at age data. Can. J. Fish. Aq. Sci. 39: 1195-1207.

Fournier, D. A., H. J. Skaug, J. Ancheta, J. Ianelli, A. Magnusson, M.N.
Maunder, A. Nielsen, and J. Sibert. 2012. AD Model Builder: using
automatic differentiation for statistical inference of highly
parameterized complex nonlinear models. Optim. Methods Softw. 27,
233-249.

Francis, R. I. C. C., 2011. Data weighting in statistical fisheries
stock assessment models. Can. J. Fish. Aquat. Sci. 68, 1124--1138.

Goethel, D. R., C. J. Rodgveller, K. B. Echave, S. K. Shotmwell, K. A.
Siwicke, D. Hanselman, P. W. Malecha, M. Cheng, M. Williams, K. Omori,
and C. R. Lunsford. 2022. Chapter 3: Assessment of the sablefish stock
in Alaska. In: Stock assessment and fishery evaluation report for the
groundfish resources of the GOA and BS/AI as projected for 2023. North
Pacific Fishery Management Council, 605 W 4th Ave, Suite 306 Anchorage,
AK 99501.

Hanselman, D. H., C. J. Rodgveller, K. H. Fenske, S. K. Shotwell, K. B.
Echave, P. W. Malecha, and C. R. Lunsford. 2018. Chapter 3: Assessment
of the sablefish stock in Alaska. In: Stock assessment and fishery
evaluation report for the groundfish resources of the GOA and BS/AI as
projected for 2019. North Pacific Fishery Management Council, 605 W 4th
Ave, Suite 306 Anchorage, AK 99501.

Hanselman, D. H., C. J. Rodgveller, C. R. Lunsford, and K. H Fenske.
2017. Chapter 3: Assessment of the sablefish stock in Alaska. In: Stock
assessment and fishery evaluation report for the groundfish resources of
the GOA and BS/AI as projected for 2018. North Pacific Fishery
Management Council, 605 W 4th Ave, Suite 306 Anchorage, AK 99501.

Heifetz, J., D. Anderl, N.E. Maloney, and T.L. Rutecki. 1999. Age
validation and analysis of ageing error from marked and recaptured
sablefish, Anoplopoma fimbria. Fish. Bull. 97: 256-263.

Johnson, S. L., and T. J. Quinn II. 1988. Catch-Age Analysis with
Auxiliary Information of sablefish in the Gulf of Alaska. Contract
report to National Marine Fisheries Service, Auke Bay, Alaska. 79
pp.~Center for Fisheries and Ocean Sciences, University of Alaska,
Juneau, Alaska.

Kimura, D. K. 1990. Approaches to age-structured separable sequential
population analysis. Can. J. Fish. Aquat. Sci. 47: 2364-2374.

Kristensen, K., A. Nielsen, C. W. Berg, H. Skaug, B. M. Bell. 2016. TMB:
Automatic Differentiation and Laplace Approximation. Journal of
Statistical Software, 70(5), 1-21.\url{doi:10.18637/jss.v070.i05}.

McAllister, M. K., Ianelli, J. N., 1997. Bayesian stock assessment using
catch-age data and the sampling: importance resampling algorithm. Can.
J. Fish. Aquat. Sci. 54,284--300.

Mueter, F. 2010. Evaluation of stock assessment and modeling options to
assess sablefish population levels and status in the Northern Southeast
Inside (NSEI) management area. Alaska Department of Fish and Game,
Special Publication No.~10-01, Anchorage, Alaska.

Sigler, M. F. 1993. Stock assessment and management of sablefish
Anoplopoma fimbria in the Gulf of Alaska. PhD Dissertation. University
of Washington. 188 pp.

Sigler, M. F., 1999. Estimation of sablefish, Anoplopoma fimbria,
abundance off Alaska with an age-structured population model. Fishery
Bulletin, 97: 591-603.

Sigler, M. F., C. R. Lunsford, J. T. Fujioka, and S. A. Lowe. 2002.
Alaska sablefish assessment for 2003. In Stock assessment and fishery
evaluation report for the groundfish fisheries of the Bering Sea and
Aleutian Islands. pp.~449-514. North Pacific Fishery Management Council,
605 W 4th Avenue, Suite 306, Anchorage, AK 99510.

Sullivan, J., B. Williams, and A. Olson. 2018. 2018 NSEI Sablefish
Assessment. State of Alaska, Department of Fish and Game, Division of
Commercial Fisheries Memorandum. June 20, 2018.

Thorson, J. T., Johnson, K. F., Methot, R. D., \& Taylor, I. G. 2017.
Model-based estimates of effective sample size in stock assessment
models using the Dirichlet-multinomial distribution. Fisheries Research,
192, 84-93.

Williams, B., and K. Van Kirk. 2017. 2017 NSEI Sablefish Assessment.
State of Alaska, Department of Fish and Game, Division of Commercial
Fisheries Memorandum. March 16, 2017.

Wood, S. N. 2011. Fast stable restricted maximum likelihood and marginal
likelihood estimation of semiparametric generalized linear models.
Journal of the Royal Statistical Society (B) 73(1):3-36.

\newpage

\hypertarget{tables}{%
\section{TABLES}\label{tables}}

\begin{longtable}[t]{lcccccc}
\caption{\label{tab:catchtab}Annual harvest objective (round lb), equal quota share (round lb), reported harvest (round lb), exvessel value, numberof permits, and effort (dats) for the directed commercial Northern Southeast Inside (NSEI) Subdistrict sablefish fishery, 1985-2022.}\\
\toprule
Year & Annual harvest objective & Equal quota share & Harvest & Exvessel value (mil) & No. of permits & No. of days\\
\midrule
1985 & 2,380,952 & NA & 2,951,056 & $2 & 105 & 3\\
1986 & 2,380,952 & NA & 3,874,269 & $2.9 & 138 & 2\\
1987 & 2,380,952 & NA & 3,861,546 & $3.4 & 158 & 1\\
1988 & 2,380,952 & NA & 4,196,601 & $4.4 & 149 & 1\\
1989 & 2,380,952 & NA & 3,767,518 & $3.5 & 151 & 1\\
\addlinespace
1990 & 2,380,952 & NA & 3,254,262 & $3.1 & 120 & 1\\
1991 & 2,380,952 & NA & 3,955,189 & $5.5 & 127 & 1\\
1992 & 2,380,952 & NA & 4,267,781 & $5.4 & 115 & 1\\
1993 & 2,380,952 & NA & 5,795,974 & $6.6 & 120 & 1\\
1994 & 4,761,905 & 38,889 & 4,708,584 & $8.1 & 121 & 30\\
\addlinespace
1995 & 4,761,905 & 38,889 & 4,543,272 & $9 & 121 & 30\\
1996 & 4,761,905 & 38,889 & 4,676,032 & $10.1 & 122 & 61\\
1997 & 4,800,000 & 39,300 & 4,752,285 & $12.2 & 122 & 76\\
1998 & 4,800,000 & 41,700 & 4,689,713 & $7.4 & 116 & 76\\
1999 & 3,120,000 & 28,000 & 3,043,272 & $6.5 & 112 & 76\\
\addlinespace
2000 & 3,120,000 & 28,600 & 3,081,797 & $8.6 & 111 & 76\\
2001 & 2,184,000 & 19,600 & 2,142,619 & $4.6 & 111 & 76\\
2002 & 2,005,000 & 18,400 & 2,009,379 & $5.3 & 109 & 76\\
2003 & 2,005,000 & 18,565 & 2,003,083 & $4.8 & 108 & 93\\
2004 & 2,245,000 & 20,787 & 2,230,396 & $4.6 & 108 & 93\\
\addlinespace
2005 & 2,053,000 & 19,400 & 2,027,187 & $5 & 106 & 93\\
2006 & 2,053,000 & 19,550 & 2,031,227 & $5.1 & 105 & 93\\
2007 & 1,488,000 & 14,500 & 1,501,483 & $3.7 & 103 & 93\\
2008 & 1,508,000 & 15,710 & 1,513,043 & $4.4 & 96 & 93\\
2009 & 1,071,000 & 12,170 & 1,069,217 & $3.3 & 88 & 93\\
\addlinespace
2010 & 1,063,000 & 12,218 & 1,054,279 & $3.8 & 87 & 93\\
2011 & 880,000 & 10,602 & 882,777 & $4.4 & 83 & 93\\
2012 & 975,000 & 12,342 & 969,775 & $3.9 & 79 & 93\\
2013 & 1,002,162 & 12,848 & 972,740 & $2.6 & 78 & 93\\
2014 & 745,774 & 9,561 & 773,534 & $2.7 & 78 & 93\\
\addlinespace
2015 & 786,748 & 10,087 & 781,702 & $3.1 & 78 & 93\\
2016 & 650,754 & 8,343 & 646,329 & $2.8 & 78 & 93\\
2017 & 720,250 & 9,234 & 714,404 & $3.6 & 78 & 93\\
2018 & 855,416 & 10,967 & 855,600 & $4.2 & 78 & 93\\
2019 & 920,093 & 11,796 & 909,341 & $4 & 78 & 93\\
\addlinespace
2020 & 1,108,003 & 14,773 & 1,101,091 & $3.1 & 75 & 93\\
2021 & 1,137,867 & 15,587 & 1,083,363 & $2.8 & 73 & 93\\
2022 & 1,233,633 & 16,899 & 1,182,518 & $3.6 & 71 & 93\\
2023 & 1,300,000 & 17,000 & NA & $NA & NA & 93\\
\bottomrule
\end{longtable}

\begin{longtable}[t]{lcc}
\caption{\label{tab:brps}Summary of key assessment results used to inform management in 2022 and 2023.  The table includes the estimates of projected biomass (sablefish aged 2 years and above) and female spawning stock biomass, estimated biological reference points of unfished female spawning biomass ($SB_{100}$), female spawning bioass at 50% of unfished levels ($SB_{50}$), and the maximum target fishing mortality of *$F_{50}$*.  Additional values include the maximum permissible Acceptable Biological Catch (max ABC) defined by *$F_{50}$*, the estimates of mortality from fishery releases that would result under max ABC and a discard mortality rate of 016, and the recommended ABC under the max 15% change management proceedure}\\
\toprule
Quantity/Status & 2022 & 2023\\
\midrule
Projected age-2 biomass (lb) & 51,885,665 & 51,975,426\\
Projected female spawning biomass (lb) & 19,714,244 & 19,836,111\\
Unfished female spawning biomass ($SB_{100}$, lb) & 28,995,917 & 28,434,171\\
Equilibrium female spawning biomass ($SB_{50}$, lb) & 14,497,958 & 14,217,085\\
max $F_{ABC}$ = $F_{50}$ & 0.062 & 0.059\\
\addlinespace
Recommended $F_{ABC}$ & 0.056 & 0.059\\
Mortality from fishery discards under max ABC (lb) & 72,190 & 69,522\\
Max ABC (lb) & 1,595,932 & 1,486,406\\
Recommended ABC (lb) & 1,443,314 & 1,573,109\\
\bottomrule
\end{longtable}

\begin{longtable}[t]{lrrrrrr}
\caption{\label{tab:decrements}Decrement types and amounts, 2017-2023.  Estimated catch in round pounds of sablefish.}\\
\toprule
Year & 2018 & 2019 & 2020 & 2021 & 2022 & 2023\\
\midrule
Acceptable biological catch & 965,354 & 1,058,037 & 1,216,743 & 1,255,056 & 1,443,314 & 1,573,109\\
Decrement Type (round lb) &  &  & Estimated Mortality &  &  & \\
Bycatch mortality in halibut fishery$^a$ & 19,583 & 18,434 & 16,207 & 38,124 & 35,406 & 35,445\\
ADF&G longline survey removal decrement (excluding catch retained by permit holders for their equal quota share$)^a$ & 15,875 & 26,260 & 24,698 & 42,499 & 95,502 & 75,636\\
Guided sport fish harvest$^b$ & 41,179 & 33,135 & 35,004 & 753 & 33,990 & 34,395\\
\addlinespace
Unguided sport fish harvest$^b$ & 5,872 & 11,340 & 5,280 & 5,631 & 9,846 & 2,655\\
Mortality from fishery deadloss$^a$ & 5,699 & 8,046 & 9,729 & 10,888 & 11,085 & 9,467\\
Mortality from fishery releases$^a$ &  & 19,142 &  &  &  & \\
Subsistence and personal use harvest$^b$ & 21,730 & 21,587 & 17,821 & 19,295 & 23,852 & 18,643\\
Total decrements & 109,938 & 137,944 & 108,740 & 117,189 & 209,681 & 177,241\\
\addlinespace
Annual harvest objective & 855,416 & 920,093 & 1,108,003 & 1,137,867 & 1,233,633 & 1,395,868\\
Permit holders & 78 & 78 & 75 & 73 & 73 & 73\\
Equal quota share & 10,967 & 11,796 & 14,773 & 15,587 & 16,899 & 19,121\\
\bottomrule
\end{longtable}

\begin{longtable}[t]{lrrrrrr}
\caption{\label{tab:modcomp}A comparison for biological reference points from candidate models in the 2023 assessment.  The base model refers to the model used in the prior assessment.  The 'tuned' model refers to the base model with age and length compositioned tuned via McAllister and Ianelli (1997) methodology.  Progression from that model adds the updated fishery selectivity curves from the federal assessment while v23 model modifications are described in this document.  The v23 no MR 5 and 10 refer to models where mark-recapture abundance estimates were dropped in the last 5 and last 10 years, respectivly,}\\
\toprule
 & Base & Tuned base & Tuned base w/ new selectivity & v23 & v23 no MR in last 5 yrs & v23 no MR in last 10 yrs\\
\midrule
Number of parameters: & 136 & 136 & 136 & 146 & 146 & 146\\
Negative log likelihood: & 2,396 & 2,396 & 6,919 & 1,799 & 1,791 & 1,747\\
Maximum gradient component: & 1.83e-06 & 1.40e-10 & 6.66e-12 & 3.32e-12 & 1.59e-07 & 1.54e-11\\
Projected age-2 biomass: & 61,145,122 & 71,385,368 & 66,676,302 & 51,975,427 & 52,424,435 & 57,371,915\\
Projected female spawning biomass: & 23,441,266 & 27,128,399 & 26,271,709 & 19,836,112 & 20,020,673 & 22,208,439\\
\addlinespace
Unfished equilibrium female spawning biomass (SPR = 100): & 30,866,389 & 32,930,727 & 33,033,358 & 28,434,171 & 28,527,598 & 30,309,066\\
Equilibrium female spawning biomass under$F_{50}$ (SPR = 50): & 15,433,194 & 16,465,363 & 16,516,679 & 14,217,086 & 14,263,799 & 15,154,533\\
Max ABC: & 1,873,598 & 2,152,761 & 1,983,085 & 1,486,406 & 1,499,490 & 1,666,358\\
Recommended ABC: & 1,659,811 & 1,659,811 & 1,659,811 & 1,486,406 & 1,499,490 & 1,659,811\\
Mortality from fishery discards under max ABC: & 79,711 & 91,383 & 82,775 & 69,522 & 70,182 & 75,426\\
\addlinespace
max $F_{ABC}$ = $F_{50}$: & 0.063 & 0.0626 & 0.059 & 0.0591 & 0.0591 & 0.059\\
F under recommended ABC: & 0.056 & 0.0483 & 0.049 & 0.0591 & 0.0591 & 0.0588\\
 &  &  &  &  &  & \\
\bottomrule
\end{longtable}

\begin{longtable}[t]{ll}
\caption{\label{tab:vardefns}Variable definitions for the statistical catch-at-age model.}\\
\toprule
Variable & Definition\\
\midrule
\addlinespace[0.3em]
\multicolumn{2}{l}{\textbf{$\textit{Indexing and model dimensions}$}}\\
\hspace{1em}$T$ & Number of years in the model\\
\hspace{1em}$t$ & Index for year in model equations\\
\hspace{1em}$A$ & Number of ages in the model\\
\hspace{1em}$a$ & Index for age in model equations\\
\hspace{1em}$a_0$ & Recruitment age (age-2)\\
\hspace{1em}$a_{+}$ & Plus group age (age-31)\\
\hspace{1em}$l$ & Index for length bin in model equations\\
\hspace{1em}$l_0$ & Recruitment length bin (41 cm)\\
\hspace{1em}$l_{+}$ & Plus group length bin (99 cm)\\
\hspace{1em}$fsh$ & NSEI longline fishery\\
\hspace{1em}$srv$ & ADF\&G longline survey\\
\hspace{1em}$MR$ & Mark-recapture abundance\\
\addlinespace[0.3em]
\multicolumn{2}{l}{\textbf{$\textit{Parameters}$}}\\
\hspace{1em}$M$ & Instantaneous natural mortality\\
\hspace{1em}$F$ & Instantaneous fishing mortality\\
\hspace{1em}$Z$ & Total instantaneous mortality\\
\hspace{1em}$S$ & Total annual survival\\
\hspace{1em}$D$ & Discard mortality\\
\hspace{1em}$s_{50}$ & Age at which 50\% of individuals are selected to the gear\\
\hspace{1em}$s_{95}$ & Age at which 95\% of individuals are selected to the gear\\
\hspace{1em}$\delta$ & Slope parameter in the logistic selectivity curve\\
\hspace{1em}$q$ & Catchability\\
\hspace{1em}$\mu_R$ & Mean log recruitment\\
\hspace{1em}$\tau_t$ & Log recruitment deviations\\
\hspace{1em}$\mu_N$ & Mean log initial numbers-at-age\\
\hspace{1em}$\psi_a$ & Log deviations of initial numbers-at-age\\
\hspace{1em}$\sigma_R$ & Variability in recruitment and initial numbers-at-age\\
\hspace{1em}$\mu_F$ & Mean log fishing mortality\\
\hspace{1em}$\phi_t$ & Log fishing mortality deviations\\
\hspace{1em}$\theta$ & Dirichlet-multinomial parameter related to effective sample size\\
\addlinespace[0.3em]
\multicolumn{2}{l}{\textbf{$\textit{Data and predicted variables}$}}\\
\hspace{1em}$w_a$ & Weight-at-age\\
\hspace{1em}$p_a$ & Proportion mature-at-age\\
\hspace{1em}$r_a$ & Proportion female-at-age\\
\hspace{1em}$R$ & Retention probability\\
\hspace{1em}$s_a$ & Selectivity-at-age\\
\hspace{1em}$\Omega_{a',a}$ & Ageing error matrix (proportion observed at age given the true age $a'$)\\
\hspace{1em}$\Lambda_{a,l,k}$ & Age-length key (proportion in length bin given age and sex)\\
\hspace{1em}$N$ & Numbers-at-age\\
\hspace{1em}$C$ & Landed catch in numbers-at-age\\
\hspace{1em}$I$, $\hat{I}$ & Indices of abundance, $\hat{I}$ are predicted values\\
\hspace{1em}$P_a$, $\hat{P}_a$ & Age compositions, $\hat{P}_a$ are predicted values\\
\hspace{1em}$P_l$, $\hat{P}_l$ & Length compositions, $\hat{P}_l$ are predicted values\\
\hspace{1em}$Y$, $\hat{Y}$ & Landed catch biomass, $\hat{Y}$ are predicted values\\
\hspace{1em}$\hat{W}$ & Estimated mortality from discards (biomass)\\
\hspace{1em}$\lambda$ & Weight for likelihood component\\
\hspace{1em}$L$ & Likelihood\\
\hspace{1em}$\omega$ & Effective sample size for age and length compositions\\
\hspace{1em}$n$ & Input sample size for Dirichlet-multinomial likelihood\\
\hspace{1em}$c$ & Small constant (0.00001)\\
\bottomrule
\end{longtable}

\begin{longtable}[t]{lrrrrrrrrr}
\caption{\label{tab:mr}A summary of data inputs to the mark-recapture models, including total individuals tagged (*K*), the total number of tags remaining once size selectivity is accounted for ($K_0$), tags not available to the longline survey or fishery (captured in ther fisheries or outside Chatham, $D_0$), recaptured individuals in the lpongline survey and fishery ($k_{srv}$ and $k_{fsh}$), number of sampled individuals in the longline survey and fishery ($n_{srv}$ and $n_{fsh}$), tags not available to the fishery (captured outside Chatham or in other fisheries during the survey, $D_{srv}$, and tags recaptred in other fisheries or outside Chatham during the fishery ($D_{fsh}$) for years with a tagging survey, 2005-2023.}\\
\toprule
Year & $K$ & $K_0$ & $D_0$ & $k_{srv}$ & $n_{srv}$ & $D_{srv}$ & $k_{fsh}$ & $n_{fsh}$ & $D_{fsh}$\\
\midrule
2005 & 7,118 & 7,118 & 9 & 0 & 0 & 104 & 690 & 180,999 & 189\\
2006 & 5,325 & 5,325 & 3 & 0 & 0 & 46 & 503 & 203,878 & 123\\
2007 & 6,158 & 6,055 & 2 & 0 & 0 & 43 & 335 & 150,729 & 77\\
2008 & 5,450 & 5,412 & 4 & 40 & 15,319 & 54 & 431 & 156,313 & 104\\
2009 & 7,071 & 7,054 & 7 & 0 & 0 & 51 & 285 & 105,709 & 92\\
\addlinespace
2010 & 7,443 & 7,307 & 4 & 54 & 14,765 & 60 & 331 & 106,201 & 38\\
2012 & 7,582 & 7,548 & 23 & 0 & 0 & 70 & 380 & 97,134 & 72\\
2013 & 7,961 & 7,921 & 24 & 0 & 0 & 89 & 374 & 99,286 & 113\\
2015 & 6,862 & 6,765 & 1 & 0 & 0 & 73 & 242 & 70,273 & 49\\
2017 & 7,096 & 6,933 & 3 & 0 & 0 & 42 & 197 & 60,409 & 11\\
\addlinespace
2018 & 9,678 & 9,160 & 13 & 0 & 0 & 77 & 183 & 65,940 & 142\\
2019 & 11,094 & 10,208 & 6 & 0 & 0 & 51 & 201 & 71,044 & 122\\
2020 & 7,916 & 7,824 & 6 & 0 & 0 & 75 & 240 & 103,190 & 129\\
2022 & 8,654 & 8,638 & 8 & 46 & 22,745 & 62 & 334 & 162,074 & 233\\
\bottomrule
\end{longtable}

\newpage

\begin{longtable}[t]{lllll}
\caption{\label{tab:fedsel}Assumed selectivity parameters for the fishery before the Equal Quota Share program started in 1994 (pre-EQS) and the fishery since the implementation of EQS for females (black points) and males (grey triangles). These parameters estimates were borrowed from the Federal stock assessment, where the Federal derby fishery and IFQ fishery were assumed to represent pre-EQS and EQS NSEI fisheries.}\\
\toprule
\multicolumn{1}{c}{ } & \multicolumn{2}{c}{Male} & \multicolumn{2}{c}{Female} \\
\cmidrule(l{3pt}r{3pt}){2-3} \cmidrule(l{3pt}r{3pt}){4-5}
 & $s_{50}$ & $\delta_{50}$ & $s_{50}$ & $\delta_{50}$\\
\midrule
Pre-EQS Fishery & 7.27 & 0.49 & 3.82 & 0.49\\
EQS Fishery & 4.29 & 0.90 & 3.34 & 1.76\\
\bottomrule
\end{longtable}
\newpage

\newpage

\begin{longtable}[t]{lcc}
\caption{\label{tab:likesum}Negative likelihood values and percent of each component to the total likelihood. The data likelihood is the sum of all likelihood contributions from data. The difference between the total likelihood and the data likelihood is the contribution of penalized likelihoods, including recruitment and fishing mortality.}\\
\toprule
Likelihood component & $NLL$ & \% of $NLL$\\
\midrule
Catch & 17.6 & 1.0\\
Fishery CPUE & 178.9 & 9.9\\
Survey CPUE & 107.7 & 6.0\\
Mark-recapture abundance & 84.9 & 4.7\\
Fishery ages & 228.9 & 12.7\\
\addlinespace
Survey ages & 274.0 & 15.2\\
Fishery lengths & 368.3 & 20.5\\
Survey lengths & 539.8 & 30.0\\
Data likelihood & 1800.1 & 100.0\\
Fishing mortality penalty & 1.4 & 0.1\\
\addlinespace
Recruitment likelihood & -11.7 & -0.6\\
SPR penalty & 0.0 & 0.0\\
Sum of catchability priors & 9.1 & 0.5\\
Total likelihood & 1798.9 & 99.9\\
\bottomrule
\end{longtable}
\begin{table}

\caption{\label{tab:keyparams}Parameter estimates from the statistical catch-at-age model. Estimates of recruitment, initial numbers-at-age, and fishing mortality deviations were excluded for brevity.}
\centering
\begin{tabular}[t]{lccc}
\toprule
Parameter & Estimate & Lower 95% CI & Upper 95% CI\\
\midrule
Survey male selectivity pre-2000, $\mbox{s}_{50}$ & 6.237 & 4.161 & 10.307\\
Survey male selectivity 2000-2022, $\mbox{s}_{50}$ & 5.042 & 4.511 & 5.685\\
Survey male selectivity pre-2000, $\delta$ & 0.562 & 0.243 & 1.300\\
Survey male selectivity 2000-2022, $\delta$ & 0.802 & 0.613 & 1.050\\
Survey female selectivity pre-2000, $\mbox{s}_{50}$ & 3.896 & 3.261 & 4.849\\
\addlinespace
Survey female selectivity 2000-2022, $\mbox{s}_{50}$ & 3.697 & 3.493 & 3.928\\
Survey female selectivity pre-2000, $\delta$ & 1.525 & 0.732 & 3.177\\
Survey female selectivity 2000-2022, $\delta$ & 2.348 & 1.649 & 3.345\\
Pre-EQS catchability, $\mbox{ln}(q_{fsh,pre-EQS})$ & -17.670 & -17.751 & -17.589\\
EQS catchability, $\mbox{ln}(q_{fsh,EQS})$ & -17.243 & -17.292 & -17.193\\
\addlinespace
Survey catchability pre-2000, $\mbox{ln}(q_{srv})$ & -16.880 & -17.003 & -16.758\\
Survey catchability 2000-2022, $\mbox{ln}(q_{srv})$ & -16.718 & -16.777 & -16.658\\
Mark-recapture catchability, $\mbox{ln}(q_{MR})$ & -0.043 & -0.062 & -0.024\\
Mean recruitment, $\mu_R$ & 799,173 & 669,879 & 953,423\\
Mean initial numbers-at-age, $\mu_N$ & 1,020,153 & 776,952 & 1,339,481\\
\addlinespace
Variability in recruitment and initial numbers-at-age (random effects parameter), $\sigma_R$ & 0.521 & 0.439 & 0.618\\
Mean fishing mortality, $\mu_F$ & 0.056 & 0.030 & 0.106\\
\bottomrule
\end{tabular}
\end{table}

\newpage

\hypertarget{figures}{%
\section{FIGURES}\label{figures}}

\begin{figure}[H]

{\centering \includegraphics[width=0.5\linewidth]{C:/Users/pjjoy/Documents/Groundfish Biometrics/Sablefish/seak_sablefish/Legacy_code_pre2023/figures/nsei_ssei_map} 

}

\caption{Northern Southeast Inside (NSEI) and Southern Southeast Inside (SSEI) Subdistricts including restricted waters of Glacier Bay National Park and Preserve and Annette Islands Reserve.}\label{fig:nseimap}
\end{figure}

\begin{figure}[H]

{\centering \includegraphics[width=0.5\linewidth]{C:/Users/pjjoy/Documents/Groundfish Biometrics/Sablefish/seak_sablefish/2023/figures/catch_exvesselvalue_2022v3} 

}

\caption{Catch, landings by port, and ex-vessel value for Northern Southeast Inside (NSEI) Subdistrict commercial sablefish 1985-2022.}\label{fig:catch}
\end{figure}

\textbackslash begin\{figure\}{[}H{]}

\{\centering \includegraphics[width=43.26in]{C:/Users/pjjoy/Documents/Groundfish Biometrics/Sablefish/seak_sablefish/2023/figures/tmb/base_vs_v23_catch_Fspr}

\}

\textbackslash caption\{Estimated catch in the NSEI fishery from 2000 -
2022 and the relationship to \(F_{40}\), \(F_{50}\) and \(F_{60}\)
(Fspr), the fishing mortality that results in a spawner-per-recruit
(SPR) of 40, 50 and 60\% of the population's virgin state, in the base
model and model v23. Note that model v23, which makes fewer assumptions
based on the federal assessment and is moredpendent on NSEI data,
estimates the population as muc closer to the ADF\&G management target
of \(F_{50}\).\}\label{fig:Fsprcomp} \textbackslash end\{figure\}

\begin{figure}[H]

{\centering \includegraphics[width=0.5\linewidth]{C:/Users/pjjoy/Documents/Groundfish Biometrics/Sablefish/seak_sablefish/2023/figures/sable_data_yprmodel_2022} 

}

\caption{A summary of the available data sources in NSEI by year.}\label{fig:datsrc}
\end{figure}

\textbackslash begin\{figure\}{[}H{]}

\{\centering \includegraphics[width=0.5\linewidth]{C:/Users/pjjoy/Documents/Groundfish Biometrics/Sablefish/seak_sablefish/2023/figures/tmb/bio_dat_2022}

\}

\textbackslash caption\{Biological inputs to the statistical
catch-at-age model, including: (A) von Bertalanffy growth model
predictions of weight-at-age (kg) by sex from the longline fishery
(black) and ADFG longline survey (grey); (B) proportion mature at age
for females estimated from the longline survey with the age at 50\%
maturity (\(a_{50}\)=6.4 yr); and (C) proportion female in the longline
survey, where the curve is the fitted line from a generalized additive
model +/- 2 standard error.\}\label{fig:bioinputs}
\textbackslash end\{figure\}

\begin{figure}[H]

{\centering \includegraphics[width=0.5\linewidth]{C:/Users/pjjoy/Documents/Groundfish Biometrics/Sablefish/seak_sablefish/2023/figures/maturity_atage_byyear_srvfem_2022} \includegraphics[width=0.5\linewidth]{C:/Users/pjjoy/Documents/Groundfish Biometrics/Sablefish/seak_sablefish/2023/figures/maturity_atlength_byyear_srvfem_2022} 

}

\caption{Changes in length- and maturity-at-age over time in the NSEI sablefish population.  There is a general trend of fish maturing at younger ages and smaller sizes.  The model uses an average of all years for the assessment.}\label{fig:deltamat}
\end{figure}

\begin{figure}[H]

{\centering \includegraphics[width=0.5\linewidth]{C:/Users/pjjoy/Documents/Groundfish Biometrics/Sablefish/seak_sablefish/2023/figures/tmb/abd_indices_2022V2} 

}

\caption{Indices of catch and abundance with the assumed error distribution, including: (A) harvest (round mt), (B) fishery catch per unit effort in round kg per hook, (C) survey catch per unit effort in number of fish per hook, and (D) mark-recapture abundance estimates in millions. The dashed vertical line in 1994 mark the transition to the Equal Quota Share program.}\label{fig:abdind}
\end{figure}

\begin{figure}[H]

{\centering \includegraphics[width=0.5\linewidth]{C:/Users/pjjoy/Documents/Groundfish Biometrics/Sablefish/seak_sablefish/2023/figures/ll_cpue_fullstand_1980_2022} 

}

\caption{CPUE in the NSEI longline sablefish fishery in round lbs per hook.  The nominal values (blue) represent values from past assessments and the fully standardized values represent the values used in this assessment.}\label{fig:fcpuestand}
\end{figure}

\begin{figure}[H]

{\centering \includegraphics[width=0.5\linewidth]{C:/Users/pjjoy/Documents/Groundfish Biometrics/Sablefish/seak_sablefish/2023/figures/tmb/agecomps_2022} 

}

\caption{Proportions-at-age for in the NSEI longline fishery (2002-2022) and ADFG longline survey (1997-2022).}\label{fig:agecomps}
\end{figure}

\begin{figure}[H]

{\centering \includegraphics[width=0.5\linewidth]{C:/Users/pjjoy/Documents/Groundfish Biometrics/Sablefish/seak_sablefish/2023/figures/raw_length_comps} 

}

\caption{Longline fishery and survey length distributions by sex from 1997-2022.}\label{fig:lencomp}
\end{figure}

\begin{figure}[H]

{\centering \includegraphics[width=0.5\linewidth]{C:/Users/pjjoy/Documents/Groundfish Biometrics/Sablefish/seak_sablefish/2023/figures/compare_comp_summaries} 

}

\caption{A comparison of the mean length and age in the longline fishery and longline survey since 1997 for male and female sablefish in the NSEI district.}\label{fig:compcomps}
\end{figure}

\begin{figure}[H]

{\centering \includegraphics[width=33.33in]{C:/Users/pjjoy/Documents/Groundfish Biometrics/Sablefish/seak_sablefish/2023/figures/tmb/retention_prob_2022} 

}

\caption{The probability of retaining a fish as a function of weight (left), sex, and age (right).}\label{fig:retention}
\end{figure}

\begin{figure}[H]

{\centering \includegraphics[width=0.5\linewidth]{C:/Users/pjjoy/Documents/Groundfish Biometrics/Sablefish/seak_sablefish/2023/figures/tmb/Final2_v23_new_2slx/selectivity_2022} 

}

\caption{Fixed age-based selectivity curves for the fishery before the Equal Quota Share program started in 1994 (pre-EQS), the fishery since the implementation of EQS, and the estimated ADFG longline survey for females (black points) and males (grey triangles) before and after the standardization of the survey in 2000. Fishery selectivity parameter estimates were borrowed from the Federal stock assessment for the derby fishery (pre-EQS) and IFQ fishery (EQS), while the survey selectivity parameters are estimated within the model.}\label{fig:slx}
\end{figure}

\begin{figure}[H]

{\centering \includegraphics[width=0.5\linewidth]{C:/Users/pjjoy/Documents/Groundfish Biometrics/Sablefish/seak_sablefish/2023/figures/tmb/Final2_v23_new_2slx/pred_abd_indices_imperial_2022} 

}

\caption{Fits to indices of catch and abundance with the assumed error distribution shown as shaded grey polygons. Input data are shown as grey points and model fits are shown in black. Indices include (A) harvest (round mt); (B) fishery catch per unit effort in round kg per hook with separate selectivity and catchability time periods before and after the implementation of the Equal Quota Share program in 1994; (C) survey catch per unit effort in number of fish per hook; and (D) mark-recapture abundance estimates in millions. Solid and dashed lines in panel D reflect years for which data were and were not available, respectively.}\label{fig:predabdind}
\end{figure}

\begin{figure}[H]

{\centering \includegraphics[width=0.75\linewidth]{C:/Users/pjjoy/Documents/Groundfish Biometrics/Sablefish/seak_sablefish/2023/figures/tmb/Final2_v23_new_2slx/Fishery_agecomps_barplot} 

}

\caption{Fits to fishery age compositions, 2002-2022. Observed and predicted proportions-at-age shown as grey bars and black lines, respectively.}\label{fig:fshage}
\end{figure}

\begin{figure}[H]

{\centering \includegraphics[width=0.75\linewidth]{C:/Users/pjjoy/Documents/Groundfish Biometrics/Sablefish/seak_sablefish/2023/figures/tmb/Final2_v23_new_2slx/Survey_agecomps_barplot} 

}

\caption{Fits to survey age compositions, 1997-2022. Observed and predicted proportions-at-age shown as grey bars and black lines, respectively.}\label{fig:srvage}
\end{figure}

\begin{figure}[H]

{\centering \includegraphics[width=0.5\linewidth]{C:/Users/pjjoy/Documents/Groundfish Biometrics/Sablefish/seak_sablefish/2023/figures/tmb/Final2_v23_new_2slx/agecomps_residplot} 

}

\caption{Standardized residuals of fits to fishery (2002-2022) and survey (1997-2022) age compositions. Size of residual scales to point size. Black points represent negative residuals (observed < predicted); white points represent positive residuals (observed > predicted).}\label{fig:residage}
\end{figure}

\begin{figure}[H]

{\centering \includegraphics[width=0.75\linewidth]{C:/Users/pjjoy/Documents/Groundfish Biometrics/Sablefish/seak_sablefish/2023/figures/tmb/Final2_v23_new_2slx/Fishery_Male_lencomps_barplot} 

}

\caption{Fits to male fishery length compositions, 2002-2022. Observed and predicted proportions-at-age shown as grey bars and black lines, respectively.}\label{fig:malefshlen}
\end{figure}

\begin{figure}[H]

{\centering \includegraphics[width=0.75\linewidth]{C:/Users/pjjoy/Documents/Groundfish Biometrics/Sablefish/seak_sablefish/2023/figures/tmb/Final2_v23_new_2slx/Fishery_Female_lencomps_barplot} 

}

\caption{Fits to female fishery length compositions, 2002-2022. Observed and predicted proportions-at-age shown as grey bars and black lines, respectively.}\label{fig:femalefshlen}
\end{figure}

\begin{figure}[H]

{\centering \includegraphics[width=0.75\linewidth]{C:/Users/pjjoy/Documents/Groundfish Biometrics/Sablefish/seak_sablefish/2023/figures/tmb/Final2_v23_new_2slx/Survey_Male_lencomps_barplot} 

}

\caption{Fits to male survey length compositions, 1997-2022. Observed and predicted proportions-at-age shown as grey bars and black lines, respectively.}\label{fig:malesrvlen}
\end{figure}

\begin{figure}[H]

{\centering \includegraphics[width=0.75\linewidth]{C:/Users/pjjoy/Documents/Groundfish Biometrics/Sablefish/seak_sablefish/2023/figures/tmb/Final2_v23_new_2slx/Survey_Female_lencomps_barplot} 

}

\caption{Fits to female survey length compositions, 1997-2022. Observed and predicted proportions-at-age shown as grey bars and black lines, respectively.}\label{fig:femalesrvlen}
\end{figure}

\begin{figure}[H]

{\centering \includegraphics[width=0.5\linewidth]{C:/Users/pjjoy/Documents/Groundfish Biometrics/Sablefish/seak_sablefish/2023/figures/tmb/Final2_v23_new_2slx/lencomps_residplot} 

}

\caption{Standardized residuals of fits to fishery (2002-2022) and survey (1997-2022) length compositions for males and females. Size of residual scales to point size. Black points represent negative residuals (observed < predicted); white points represent positive residuals (observed > predicted).}\label{fig:residlen}
\end{figure}

\begin{figure}[H]

{\centering \includegraphics[width=0.5\linewidth]{C:/Users/pjjoy/Documents/Groundfish Biometrics/Sablefish/seak_sablefish/2023/figures/tmb/Final2_v23_new_2slx/derived_ts_imperial_2022} 

}

\caption{Model predictions of (A) age-2 recruitment (millions), (B) female spawning stack biomass (million lb), (C) exploitable abundance (millions), and (D) exploitable biomass (million lb).}\label{fig:derivedts}
\end{figure}

\begin{figure}[H]

{\centering \includegraphics[width=0.5\linewidth]{C:/Users/pjjoy/Documents/Groundfish Biometrics/Sablefish/seak_sablefish/2023/figures/tmb/Final2_v23_new_2slx/fishing_mort} 

}

\caption{Model-estimated fishing mortality rate (top) and realized harvest rate (bottom), defined as the ratio of total predicted catch to exploitable biomass. Total predicted catch is the sum of landed catch and discarded biomass assumed to die post-release.}\label{fig:fishmort}
\end{figure}

\begin{figure}[H]

{\centering \includegraphics[width=0.5\linewidth]{C:/Users/pjjoy/Documents/Groundfish Biometrics/Sablefish/seak_sablefish/2023/output/tmb/retrospective_v23_new_2slx/retrospective_spawn_biom_2022} 

}

\caption{Mohn's $\rho$ and retrospective peels of sablefish spawning biomass for the last 9 years.}\label{fig:mohnsbiom}
\end{figure}

\begin{figure}[H]

{\centering \includegraphics[width=0.5\linewidth]{C:/Users/pjjoy/Documents/Groundfish Biometrics/Sablefish/seak_sablefish/2023/output/tmb/retrospective_v23_new_2slx/retrospective_recruitment_2022} 

}

\caption{Mohn's $\rho$ and retrospective peels of sablefish recruitment for the last 9 years.}\label{fig:mohnsrec}
\end{figure}

\#```\{r ageerror, fig.cap= paste0(``Ageing error matrix used in the
model, showing the probability of \#observing an age given the true
\#age (Heifetz et al.~1999).'')\}

\#knitr::include\_graphics(paste0(``../figures/tmb/ageing\_error.png''))
\#```

\#```\{r srvlen, fig.cap= paste0(``Longline survey length distributions
by sex from 2012-'',YEAR,``. Vertical \#bars represent the data nd the
black line shows the length composition from the most recent year
\#superimposed on previous years.'')\}

\#knitr::include\_graphics(paste0(figures\_dir,``/llsrv\_lencomps\_2012\_'',YEAR,``.png''),
rel\_path=FALSE) \#```

\#```\{r agelenkey, fig.cap= paste0(``Age-length key used in the model,
with the relative size of the \#bubbles reflecting the probability
\#that a fish of a given age falls within a certain length bin (Echave
\#et al.~2012). The probabilities sum to 1 across each age.'')\}

\#knitr::include\_graphics(paste0(``../figures/tmb/age\_length\_key.png''))
\#```

\#```\{r residabdind, fig.cap= paste0(``Standardized residuals of fits
to indices of catch and abundance, \#including: (A) harvest, (B)
\#fishery catch per unit effort, (C) survey catch per unit effort, and
(D) \#mark-recapture (MR) abundance.'')\}

\#knitr::include\_graphics(paste0(tmb\_mod\_figures\_dir,``/selectivity\_'',YEAR,``.png''))
\#knitr::include\_graphics(paste0(``../figures/tmb/presid\_abd\_indices.png''))
\#```

\end{document}
